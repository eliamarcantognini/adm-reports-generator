\documentclass[12pt]{article}
\usepackage[a4paper, top=2cm, bottom=3cm, left=1cm, right=1cm]{geometry}
\usepackage[utf8]{inputenc}
\usepackage{graphicx}
\usepackage{caption}
\usepackage{multirow}
\usepackage{multicol}
\usepackage{makecell}
\usepackage{tabularx}
\usepackage{amssymb}
\usepackage{parskip}
\usepackage{fancyhdr}
\usepackage{enumitem}
\usepackage{lastpage}

\newcolumntype{Y}{>{\centering\arraybackslash}X}

\newcommand\denominazioneEsercizio{$denominazioneEsercizio}
\newcommand\indirizzoEsercizio{$indirizzoEsercizio}
\newcommand\picfAttivita{$picfAttivita}
\newcommand\denominazioneEsercente{$denominazioneEsercente}
\newcommand\verbalizzanteuno{$verbalizzante1}
\newcommand\verbalizzantedue{$verbalizzante2}
\newcommand\verbalizzantetre{$verbalizzante3}
\newcommand\ordineDiServizio{$ordineDiServizio}
\newcommand\dataVerifica{$dataVerifica}
\newcommand\nRivendita{$nRivendita}
\newcommand\nLotto{$nLotto}
\newcommand\comuneEsercizio{$comuneEsercizio}

\newcommand\signature{%
  \par\vspace{8ex}\noindent
  \begin{tabular}[t]{@{}c@{}}
    I VERBALIZZANTI\\ \\
    \makebox[15em]{\dotfill}\\
    \\
    \makebox[15em]{\dotfill}\\
    \\
    \makebox[15em]{\dotfill}
  \end{tabular}
  \hfill
  \begin{tabular}[t]{@{}c@{}}
    LA PARTE\\ \\
    \makebox[15em]{\dotfill}
  \end{tabular}
}

\pagestyle{fancy}
\fancyhf{}
\setlength{\headheight}{1.5cm}
\lhead{\includegraphics[height=1cm]{logo-adm.jpg}}
% DA CAMBIARE
\rhead{\scriptsize{Segue processo verbale di constatazione nei confronti della}\\\scriptsize{rivendita n. \nRivendita in \comuneEsercizio}\\\scriptsize{redatto in data \dataVerifica}}
\cfoot{Pagina \thepage \hspace{1pt} di \pageref{LastPage}}


\begin{document}
%%%%%%%%%%%%%%%%%%%%%%%%%%%%%%%%%%%%
%%%%%%%%%%%%%%%%%%%%%%%%%%%%%%%%%%%%
%               FRONTESPIZIO
%%%%%%%%%%%%%%%%%%%%%%%%%%%%%%%%%%%%
%%%%%%%%%%%%%%%%%%%%%%%%%%%%%%%%%%%%
\thispagestyle{empty}

\begin{figure}[h]
    \centering
    \includegraphics{logo-adm.jpg}
    \caption*{Ufficio dei Monopoli per le Marche\\Sezione distaccata di Pesaro}
    \label{fig:logoadm}
\end{figure}

{\centering
    \textbf{PROCESSO VERBALE DI CONSTATAZIONE - ACCERTAMENTO}
    \par
}

\begin{tabularx}{\linewidth}{|c|X|}
   \hline
    \multirowcell{10}{DATI \\ ESERCIZIO} & \\ & Denominazione: \denominazioneEsercizio \\ & Indirizzo: \indirizzoEsercizio \\ &  P.IVA/C.F.: \picfAttivita \\ & Titolare: \denominazioneEsercente \\& Telefono: \_\_\_\_\_\_\_\_\_\_\_\_\_\_\_\_\_\_\_\_\_\_\_\_\_\_\_\_\_\_\_\_\_\_\_\_\_\_\_\_\_\_\_\_\_\_\_\_\_\_\_\_\_\_\_\_\_\_\_\_\_\_\_\_\_\_\_\_\_\_\_\_\_\_\_\_\_\_\_\_\_\_ \\ & Orari: \_\_\_\_\_\_\_\_\_\_\_\_\_\_\_\_\_\_\_\_\_\_\_\_\_\_\_\_\_\_\_\_\_\_\_\_\_\_\_\_\_\_\_\_\_\_\_\_\_\_\_\_\_\_\_\_\_\_\_\_\_\_\_\_\_\_\_\_\_\_\_\_\_\_\_\_\_\_\_\_\_\_\_\_\_\_ \\ & Apertura domenicale: \_\_\_\_\_\_\_\_\_\_\_\_\_\_\_\_\_\_\_\_\_\_\_\_\_\_\_\_\_\_\_\_\_\_\_\_\_\_\_\_\_\_\_\_\_\_\_\_\_\_\_\_\_\_\_\_\_\_\_\_\_\_\_\_\_\_\_ \\ & Giorno di chiusura: \_\_\_\_\_\_\_\_\_\_\_\_\_\_\_\_\_\_\_\_\_\_\_\_\_\_\_\_\_\_\_\_\_\_\_\_\_\_\_\_\_\_\_\_\_\_\_\_\_\_\_\_\_\_\_\_\_\_\_\_\_\_\_\_\_\_\_\_\_\_ \\ & \\
    \hline
    \multirowcell{4}{OGGETTO\\DELLA\\VERIFICA} & \\ & RIVENDITA N. \nRivendita \\ & RICEVITORIA LOTTO N. \nLotto  \\ & \\
    \hline
    \multirowcell{10}{ANAGRAFICA\\ DELLA \\ PARTE} & \\ & COGNOME: \_\_\_\_\_\_\_\_\_\_\_\_\_\_\_\_\_\_\_\_\_\_\_\_\_\_\_\_\_\_\_\_ NOME: \_\_\_\_\_\_\_\_\_\_\_\_\_\_\_\_\_\_\_\_\_\_\_\_\_\_\_\_\_\_\_\_\_\_ \\ & NATO A: \_\_\_\_\_\_\_\_\_\_\_\_\_\_\_\_\_\_\_\_\_\_\_\_\_\_\_\_\_\_\_\_\_\_\_\_\_\_\_\_\_\_\_\_\_\_\_\_\_\_\_\_ IL: \_\_\_\_\_\_\_\_\_\_\_\_\_\_\_\_\_\_\_\_\_\_\_\_ \\ & RESIDENTE IN: \_\_\_\_\_\_\_\_\_\_\_\_\_\_\_\_\_\_\_\_\_\_\_\_\_\_\_\_\_\_\_\_\_\_\_\_\_\_\_\_\_\_\_\_\_\_\_\_\_\_\_\_\_\_\_\_\_\_\_\_\_\_\_\_\_\_\_\_\_\_\_\_ \\ & INDIRIZZO: \_\_\_\_\_\_\_\_\_\_\_\_\_\_\_\_\_\_\_\_\_\_\_\_\_\_\_\_\_\_\_\_\_\_\_\_\_\_\_\_\_\_\_\_\_\_\_\_\_\_\_\_\_\_\_\_\_\_\_\_\_\_\_\_\_\_\_\_\_\_\_\_\_\_\_\_\_\_ \\ & Identificato a mezzo \begin{math}\;\square\end{math} carta d'identità \begin{math}\;\square\end{math} patente di guida \begin{math}\;\square\end{math} altro \\ & rilasciato/a da \_\_\_\_\_\_\_\_\_\_\_\_\_\_\_\_\_\_\_\_\_\_\_\_\_\_\_\_\_\_\_\_\_\_\_\_\_\_\_\_\_\_\_ scadente il \_\_\_\_\_\_\_\_\_\_\_\_\_\_\_\_\_\_ \\ & IN QUALITÀ DI: \\ &\begin{math}\square\end{math} titolare/legale rappresentante  \begin{math}\square\end{math} dipendente \begin{math}\square\end{math} Altro\_\_\_\_\_\_\_\_\_\_\_\_\_\_\_\_\_\_\_\_\_\_\_\_\_\_\_\_\_ \\ & \\
    \hline
    \multirowcell{7}{AGENZIA DELLE\\DOGANE\\E DEI MONOPOLI\\ \scriptsize{Ufficio dei Monopoli per}\\ \scriptsize{le Marche}\\ \scriptsize{Sede di Pesaro}} & \\ & Verbalizzante: \verbalizzanteuno \\ & \\ & Verbalizzante: \verbalizzantedue \\ & \\ & Verbalizzante: \verbalizzantetre  \\ & \\
    \hline
\end{tabularx}


Data verifica: \dataVerifica \\
Inizio verifica: \_\_\_\_\_\_.\_\_\_\_\_\_.
\newpage
%%%%%%%%%%%%%%%%%%%%%%%%%%%%%%%%%%%%
%%%%%%%%%%%%%%%%%%%%%%%%%%%%%%%%%%%%
%               CONTENUTO
%%%%%%%%%%%%%%%%%%%%%%%%%%%%%%%%%%%%
%%%%%%%%%%%%%%%%%%%%%%%%%%%%%%%%%%%%


In data odierna \dataVerifica, alle ore \_\_\_\_\_.\_\_\_\_\_, i sopracitati dipendenti, funzionari dell'Agenzia delle Dogane e dei Monopoli – Ufficio dei Monopoli per le Marche, Sede di Pesaro, hanno eseguito un sopralluogo presso la rivendita sopra indicata, nell’ambito delle prerogative di cui all’art. 79 del D.P.R. 1074/58, in esecuzione dell’ordine di servizio \ordineDiServizio.


Dopo aver dimostrato la propria identità mediante esibizione delle tessere di servizio i funzionari ADM hanno reso edotta la parte sullo scopo della visita e precisamente: accertare che la rivendita venga gestita in conformità
\begin{multicols}{2}
\begin{itemize}
    \item alla Legge 1293/57;
    \item al Regolamento approvato con Decreto del Presidente della Repubblica 1074/58;
    \item al Capitolato d'oneri;
    \item al divieto di vendita dei tabacchi ai minori di anni diciotto;
    \item alle altre disposizioni emanate dall'Agenzia delle Dogane e dei Monopoli.
\end{itemize}
\end{multicols}

\textbf{Viene verificato quanto segue:}

\begin{tabularx}{\textwidth}{|p{15cm}|Y|Y|}
    \hline
    1. la rivendita è regolarmente funzionante nei locali autorizzati dall’Agenzia (art. 78 del D.P.R. 1074/58) & SÌ & NO \\
    \hline
    2. il personale presente in rivendita è regolarmente autorizzato (art. 64 del D.P.R. 1074/58) & SÌ & NO \\
    \hline
    3.	la rivendita è rifornita di tabacchi lavorati in quantità adeguata alle esigenze di consumo (art. 15 lettera a del capitolato d’oneri) & SÌ & NO \\
    \hline
    4.	la rivendita è rifornita di  biglietti delle lotterie nazionali ad estrazione differita (art. 15 lettera b del capitolato d’oneri) & SÌ & NO \\
    \hline
    5.	la mostra dei tabacchi è adeguata e garantisce la neutralità dell’offerta (art. 9 del capitolato d’oneri) & SÌ & NO \\
    \hline
    6.	i tabacchi lavorati sono venduti nei condizionamenti usuali presenti in commercio (art. 15 del capitolato d’oneri) & SÌ & NO \\
    \hline
    7.	i generi di monopolio sono custoditi e venduti esclusivamente nei locali e nei luoghi per i quali è il rivenditore è stato autorizzato (art 7 del capitolato d’oneri) & SÌ & NO \\
    \hline
    8.	la vendita dei tabacchi non è associata o condizionata alla consegna all’acquirente di gadget e omaggi  e non sono esposti prodotti, materiali od oggettistica di qualsiasi genere che, anche in modo indiretto, costituiscano richiamo a qualsiasi marchio di prodotto da fumo (art. 17, lettera b e d, del capitolato d’oneri) & SÌ & NO \\
    \hline
    9.	sono tenute in vista all’interno della rivendita la licenza di esercizio e la tariffa dei prezzi di vendita dei generi di monopolio (art. 9 del capitolato d’oneri) & SÌ & NO \\
    \hline
    10.	il distributore automatico di sigarette è dotato di sistema automatico di rilevamento dell’età anagrafica dell’acquirente ed eroga prodotti tramite lettura  automatica dei documenti anagrafici rilasciati dalla  pubblica amministrazione (come previsto dalla vigente normativa) & SÌ & NO \\
    \hline
    11.	la vendita di valori postali e bollati è regolare (art 15 del capitolato d’oneri) & SÌ & NO \\
    \hline
    12.	l’insegna fuori del locale è esposta in posizione ben visibile al pubblico (art. 9 del capitolato d’oneri) & SÌ & NO \\
    \hline
    13.	i generi di monopolio pervengono direttamente, con regolare bolletta, dall’organo di rifornimento prescritto dall’Agenzia (art. 16, lettera d, del capitolato d’oneri) & SÌ & NO \\
    \hline
    14.	la rivendita rifornisce il patentino n. \_\_\_\_\_\_\_\_\_\_\_ come da copie bollettario Mod U/88/Pat regolarmente custodite (art. 14 del capitolato d’oneri) & SÌ & NO \\
    \hline
\end{tabularx}
\begin{tabularx}{\textwidth}{|p{15cm}|Y|Y|}
    \hline
    15. gli scontrini vincenti e pagati , validati manualmente, annullati, e prenotati del lotto, 10elotto e Millionday degli ultimi sei mesi (determina n. 45593 del 04/05/17) sono regolarmente custoditi presso la rivendita & SÌ & NO \\
    \hline
    16. all’atto dell’accesso  non è stata riscontrata la violazione del divieto di vendita dei tabacchi a minori di anni 18 & SÌ & NO \\
    \hline
    17. all’atto dell’accesso  non è  stata riscontrata la violazione del divieto di partecipazione ai giochi con vincita in denaro ai minori di anni 18& SÌ & NO \\
    \hline
    18. sono regolarmente esposti i cartelli di divieto della vendita di tabacchi ai minori di anni 18 & SÌ & NO \\
    \hline
    19. sono regolarmente esposti i cartelli di divieto di partecipazione ai giochi con vincita in denaro ai minori di anni 18 come prescritto dal comma 5 art. 7 del D.L. n. 158/2012 & SÌ & NO \\
    \hline
    20. Sono presenti prodotti accessori da fumo & SÌ & NO \\
    \hline
    20bis. Si allega copia documento di acquisto dei prodotti accessori da fumo& SÌ & NO \\
    \hline

\end{tabularx}

\paragraph{VERIFICA DEL LOTTO}

\begin{itemize}
    \item Settimana contabile lotto controllata dal \_\_\_\_\_\_/\_\_\_\_\_\_/\_\_\_\_\_\_\_\_\_\_ al \_\_\_\_\_\_/\_\_\_\_\_\_/\_\_\_\_\_\_\_\_\_\_.
    \\Verificati:
    \begin{itemize}
        \item n. \_\_\_\_\_\_\_\_\_ scontrini vincenti
        \item n. \_\_\_\_\_\_\_\_\_ scontrini vincenti validati manualmente
        \item n. \_\_\_\_\_\_\_\_\_ scontrini annullati
        \item n. \_\_\_\_\_\_\_\_\_ scontrini prenotati
    \end{itemize}
    \item Settimana/giornata contabile 10 e lotto controllata dal \_\_\_\_\_\_/\_\_\_\_\_\_/\_\_\_\_\_\_\_\_\_\_ al \_\_\_\_\_\_/\_\_\_\_\_\_/\_\_\_\_\_\_\_\_\_\_.
    \\Verificati:
    \begin{itemize}
        \item n. \_\_\_\_\_\_\_\_\_ scontrini vincenti
        \item n. \_\_\_\_\_\_\_\_\_ scontrini vincenti validati manualmente
        \item n. \_\_\_\_\_\_\_\_\_ scontrini annullati
        \item n. \_\_\_\_\_\_\_\_\_ scontrini prenotati
    \end{itemize}
    \item Settimana/giornata contabile Millionday controllata dal \_\_\_\_\_\_/\_\_\_\_\_\_/\_\_\_\_\_\_\_\_\_\_ al \_\_\_\_\_\_/\_\_\_\_\_\_/\_\_\_\_\_\_\_\_\_\_.
    \\Verificati:
    \begin{itemize}
        \item n. \_\_\_\_\_\_\_\_\_ scontrini vincenti
        \item n. \_\_\_\_\_\_\_\_\_ scontrini vincenti validati manualmente
        \item n. \_\_\_\_\_\_\_\_\_ scontrini annullati
        \item n. \_\_\_\_\_\_\_\_\_ scontrini prenotati
    \end{itemize}
    \item nella ricevitoria sono in funzione n. \_\_\_\_\_\_\_\_\_ terminali lotto;
    \item personale autorizzato ai sensi art. 28 L. 1293/57 - presente al momento della verifica: \\ \_\_\_\_\_\_\_\_\_\_\_\_\_\_\_\_\_\_\_\_\_\_\_\_\_\_\_\_\_\_\_\_\_\_\_\_\_\_\_\_\_\_\_\_\_\_\_\_\_\_\_\_\_\_\_\_\_\_\_\_\_\_\_\_\_\_\_\_\_\_\_\_\_\_\_\_\_\_\_\_\_\_\_\_\_\_\_\_\_\_\_\_\_\_\_\_\_\_\_\_\_\_\_\_\_\_\_\_\_\_\_\_\_\_\_\_\_\_\_\_ \\ \_\_\_\_\_\_\_\_\_\_\_\_\_\_\_\_\_\_\_\_\_\_\_\_\_\_\_\_\_\_\_\_\_\_\_\_\_\_\_\_\_\_\_\_\_\_\_\_\_\_\_\_\_\_\_\_\_\_\_\_\_\_\_\_\_\_\_\_\_\_\_\_\_\_\_\_\_\_\_\_\_\_\_\_\_\_\_\_\_\_\_\_\_\_\_\_\_\_\_\_\_\_\_\_\_\_\_\_\_\_\_\_\_\_\_\_\_\_\_\_
\end{itemize}

\paragraph{VERIFICA AUTENTICITÀ DEI CONTRASSEGNI} In ottemperanza alle circolari, prot. n. DAC/CTL/11405/2012 del 26/07/2012 e prot. n. DAC/DIR/868/2012 del 16/11/2012, i verbalizzanti hanno effettuato le sotto indicate prove, finalizzate ad accertare l’autenticità dei contrassegni di Stato di cui al D. D. 23/06/2011, utilizzando il lettore \textbf{HD20}, matricola \textbf{PAG 521}:

\begin{tabularx}{\textwidth}{|p{.15\textwidth}|p{.45\textwidth}|p{.1\textwidth}|Y|}
    \hline
    Codice & Denominazione & n. pacchetti & Esito (regolare, irregolare, impossibile per presenza vecchio contrassegno)  \\
    \hline
     & & & \\[20pt]
    \hline
     & & & \\[20pt]
    \hline
     & & & \\[20pt]
    \hline
     & & & \\[20pt]
    \hline
     & & & \\[20pt]
    \hline
     & & & \\[20pt]
    \hline
\end{tabularx}

Vista l’irregolarità riscontrata su n. \_\_\_\_\_\_\_\_\_\_\_\_ confezioni di \_\_\_\_\_\_\_\_\_\_\_\_\_\_\_\_\_\_\_\_\_\_\_\_\_\_\_\_\_\_\_\_\_\_\_\_\_\_\_\_\_\_\_\_\_\_\_\_\_\_\_\_\_ marca \_\_\_\_\_\_\_\_\_\_\_\_\_\_\_\_\_\_\_\_\_\_\_\_\_\_\_\_\_\_\_\_\_\_\_\_\_\_\_\_\_\_\_\_\_\_\_\_ codice  \_\_\_\_\_\_\_\_\_\_\_\_\_\_\_\_\_\_\_\_\_\_\_\_ i verbalizzanti hanno provveduto a dare immediata comunicazione  del fatto alla Guardia di Finanza con richiesta di intervento tramite il n. 117.  In esito alla richiesta la Guardia di Finanza \\
\begin{math}\square\end{math} È \begin{math}\square\end{math} NON È\\
intervenuta ed ha emesso apposito verbale  di \_\_\_\_\_\_\_\_\_\_\_\_\_\_\_\_\_\_\_\_\_\_\_\_\_\_\_\_\_\_\_\_\_\_\_\_\_\_\_\_\_\_\_\_\_\_\_\_\_\_\_\_\_\_\_\_\_\_\_\_\_\_\_\_\_\_\_\_\_\_\_\_\_\_.



\paragraph{LOTTERIE AD ESTRAZIONE ISTANTANEA}
Ai fini delle attività di controllo, previste dall’\textbf{art. 25} della \textit{Convenzione per il rapporto di concessione dell’esercizio dei giochi pubblici denominati lotterie nazionali ad estrazione istantanea} nonché dall’\textbf{art. 11} del \textit{Contratto di autorizzazione all’esercizio dell’attività di vendita dei biglietti delle lotterie nazionali ad estrazione istantanee}, i verbalizzanti\\
\begin{math}\square\end{math} HANNO \begin{math}\square\end{math} NON HANNO\\
constatato che all’interno dei locali del predetto esercizio sono posti in vendita biglietti delle \textbf{lotterie ad estrazione istantanea}.\\
L'esame visivo dei suddetti biglietti posti in vendita risulta:
\begin{itemize}[label={\begin{math}\square\end{math}}]
    \item REGOLARE: biglietti integri, non contraffatti, relativi a concorsi autorizzati e non dichiarati cessati;
    \item NON REGOLARE: viene redatto il \textit{Verbale A)} di constatazione (allegato);\\ viene redatto il \textit{Verbale B)} di constatazione e comunicazione di notizia di reato ai sensi dell'art. 331 c.p.p. (allegato).\\
    Nel caso di riscontrata irregolarità, si ordina al titolare del punto vendita di ritirare immediatamente dalla vendita i biglietti di cui all’allegato Verbale A) – B) e custodirli al fine di consentire l’effettuazione delle necessarie verifiche da parte del concessionario e/o dell’organo giudiziario competente.
\end{itemize}

\paragraph{GIOCHI NUMERICI A TOTALIZZATORE NAZIONALE}
In ottemperanza alla circolare prot. n. 19039 del 12/02/2018 è stata controllata l’esistenza di terminali per giochi numerici a totalizzatore nazionale (GNTN) ed efficienza operativa degli stessi:
\begin{multicols}{2}
    \begin{itemize}
        \item terminali presenti: \begin{math}\square\end{math} SÌ \begin{math}\square\end{math} NO
        \item terminali operativi: \begin{math}\square\end{math} SÌ \begin{math}\square\end{math} NO
    \end{itemize}
\end{multicols}

\paragraph{VERIFICA SUCCEDANEI DEL TABACCO} Fatto divieto di vendita di prodotti o sostanze atte a surrogare i tabacchi, salvo deroga per i liquidi da inalazione, secondo l'Art. 16 del Capitolato d'oneri(obblighi del rivenditore derivanti dal rapporto di concessione). Si verifica:
\begin{itemize}
    \item Disponibilità di prodotti atti a surrogare i tabacchi: \begin{math}\square\end{math} SÌ \begin{math}\square\end{math} NO
    \item Riscontrata presenza di cannabis light in foglia, infiorescenza, olio o resina \begin{math}\square\end{math} SÌ \begin{math}\square\end{math} NO
\end{itemize}



Le eventuali irregolarità sopra accertate costituiscono violazione sanzionabile a norma dell’art. 7 della Legge 8 novembre 2012 n. 189 e della Legge n. 1293/1957 che saranno oggetto di successiva contestazione nei termini previsti dalle disposizioni vigenti.
\begin{multicols}{2}
    \begin{math}\square\end{math} REGOLARE \\
    \begin{math}\square\end{math} IRREGOLARE
\end{multicols}








\begin{center}
    \textbf{Dichiarazioni della parte}
\end{center}

\_\_\_\_\_\_\_\_\_\_\_\_\_\_\_\_\_\_\_\_\_\_\_\_\_\_\_\_\_\_\_\_\_\_\_\_\_\_\_\_\_\_\_\_\_\_\_\_\_\_\_\_\_\_\_\_\_\_\_\_\_\_\_\_\_\_\_\_\_\_\_\_\_\_\_\_\_\_\_\_\_\_\_\_\_\_\_\_\_\_\_\_\_\_\_\_\_\_\_\_\_\_\_\_\_\_\_\_\_\_\_\_\_\_\_\_\_\_\_\_\_\_\_\_\_\_\_ \\
\_\_\_\_\_\_\_\_\_\_\_\_\_\_\_\_\_\_\_\_\_\_\_\_\_\_\_\_\_\_\_\_\_\_\_\_\_\_\_\_\_\_\_\_\_\_\_\_\_\_\_\_\_\_\_\_\_\_\_\_\_\_\_\_\_\_\_\_\_\_\_\_\_\_\_\_\_\_\_\_\_\_\_\_\_\_\_\_\_\_\_\_\_\_\_\_\_\_\_\_\_\_\_\_\_\_\_\_\_\_\_\_\_\_\_\_\_\_\_\_\_\_\_\_\_\_\_ \\\_\_\_\_\_\_\_\_\_\_\_\_\_\_\_\_\_\_\_\_\_\_\_\_\_\_\_\_\_\_\_\_\_\_\_\_\_\_\_\_\_\_\_\_\_\_\_\_\_\_\_\_\_\_\_\_\_\_\_\_\_\_\_\_\_\_\_\_\_\_\_\_\_\_\_\_\_\_\_\_\_\_\_\_\_\_\_\_\_\_\_\_\_\_\_\_\_\_\_\_\_\_\_\_\_\_\_\_\_\_\_\_\_\_\_\_\_\_\_\_\_\_\_\_\_\_\_ \\\_\_\_\_\_\_\_\_\_\_\_\_\_\_\_\_\_\_\_\_\_\_\_\_\_\_\_\_\_\_\_\_\_\_\_\_\_\_\_\_\_\_\_\_\_\_\_\_\_\_\_\_\_\_\_\_\_\_\_\_\_\_\_\_\_\_\_\_\_\_\_\_\_\_\_\_\_\_\_\_\_\_\_\_\_\_\_\_\_\_\_\_\_\_\_\_\_\_\_\_\_\_\_\_\_\_\_\_\_\_\_\_\_\_\_\_\_\_\_\_\_\_\_\_\_\_\_ \\\_\_\_\_\_\_\_\_\_\_\_\_\_\_\_\_\_\_\_\_\_\_\_\_\_\_\_\_\_\_\_\_\_\_\_\_\_\_\_\_\_\_\_\_\_\_\_\_\_\_\_\_\_\_\_\_\_\_\_\_\_\_\_\_\_\_\_\_\_\_\_\_\_\_\_\_\_\_\_\_\_\_\_\_\_\_\_\_\_\_\_\_\_\_\_\_\_\_\_\_\_\_\_\_\_\_\_\_\_\_\_\_\_\_\_\_\_\_\_\_\_\_\_\_\_\_\_ \\\_\_\_\_\_\_\_\_\_\_\_\_\_\_\_\_\_\_\_\_\_\_\_\_\_\_\_\_\_\_\_\_\_\_\_\_\_\_\_\_\_\_\_\_\_\_\_\_\_\_\_\_\_\_\_\_\_\_\_\_\_\_\_\_\_\_\_\_\_\_\_\_\_\_\_\_\_\_\_\_\_\_\_\_\_\_\_\_\_\_\_\_\_\_\_\_\_\_\_\_\_\_\_\_\_\_\_\_\_\_\_\_\_\_\_\_\_\_\_\_\_\_\_\_\_\_\_ \\\_\_\_\_\_\_\_\_\_\_\_\_\_\_\_\_\_\_\_\_\_\_\_\_\_\_\_\_\_\_\_\_\_\_\_\_\_\_\_\_\_\_\_\_\_\_\_\_\_\_\_\_\_\_\_\_\_\_\_\_\_\_\_\_\_\_\_\_\_\_\_\_\_\_\_\_\_\_\_\_\_\_\_\_\_\_\_\_\_\_\_\_\_\_\_\_\_\_\_\_\_\_\_\_\_\_\_\_\_\_\_\_\_\_\_\_\_\_\_\_\_\_\_\_\_\_\_ \\\_\_\_\_\_\_\_\_\_\_\_\_\_\_\_\_\_\_\_\_\_\_\_\_\_\_\_\_\_\_\_\_\_\_\_\_\_\_\_\_\_\_\_\_\_\_\_\_\_\_\_\_\_\_\_\_\_\_\_\_\_\_\_\_\_\_\_\_\_\_\_\_\_\_\_\_\_\_\_\_\_\_\_\_\_\_\_\_\_\_\_\_\_\_\_\_\_\_\_\_\_\_\_\_\_\_\_\_\_\_\_\_\_\_\_\_\_\_\_\_\_\_\_\_\_\_\_ \\\_\_\_\_\_\_\_\_\_\_\_\_\_\_\_\_\_\_\_\_\_\_\_\_\_\_\_\_\_\_\_\_\_\_\_\_\_\_\_\_\_\_\_\_\_\_\_\_\_\_\_\_\_\_\_\_\_\_\_\_\_\_\_\_\_\_\_\_\_\_\_\_\_\_\_\_\_\_\_\_\_\_\_\_\_\_\_\_\_\_\_\_\_\_\_\_\_\_\_\_\_\_\_\_\_\_\_\_\_\_\_\_\_\_\_\_\_\_\_\_\_\_\_\_\_\_\_ \\\_\_\_\_\_\_\_\_\_\_\_\_\_\_\_\_\_\_\_\_\_\_\_\_\_\_\_\_\_\_\_\_\_\_\_\_\_\_\_\_\_\_\_\_\_\_\_\_\_\_\_\_\_\_\_\_\_\_\_\_\_\_\_\_\_\_\_\_\_\_\_\_\_\_\_\_\_\_\_\_\_\_\_\_\_\_\_\_\_\_\_\_\_\_\_\_\_\_\_\_\_\_\_\_\_\_\_\_\_\_\_\_\_\_\_\_\_\_\_\_\_\_\_\_\_\_\_

\begin{center}
    \textbf{Dichiarazioni dei verbalizzanti}
\end{center}

\_\_\_\_\_\_\_\_\_\_\_\_\_\_\_\_\_\_\_\_\_\_\_\_\_\_\_\_\_\_\_\_\_\_\_\_\_\_\_\_\_\_\_\_\_\_\_\_\_\_\_\_\_\_\_\_\_\_\_\_\_\_\_\_\_\_\_\_\_\_\_\_\_\_\_\_\_\_\_\_\_\_\_\_\_\_\_\_\_\_\_\_\_\_\_\_\_\_\_\_\_\_\_\_\_\_\_\_\_\_\_\_\_\_\_\_\_\_\_\_\_\_\_\_\_\_\_ \\
\_\_\_\_\_\_\_\_\_\_\_\_\_\_\_\_\_\_\_\_\_\_\_\_\_\_\_\_\_\_\_\_\_\_\_\_\_\_\_\_\_\_\_\_\_\_\_\_\_\_\_\_\_\_\_\_\_\_\_\_\_\_\_\_\_\_\_\_\_\_\_\_\_\_\_\_\_\_\_\_\_\_\_\_\_\_\_\_\_\_\_\_\_\_\_\_\_\_\_\_\_\_\_\_\_\_\_\_\_\_\_\_\_\_\_\_\_\_\_\_\_\_\_\_\_\_\_ \\\_\_\_\_\_\_\_\_\_\_\_\_\_\_\_\_\_\_\_\_\_\_\_\_\_\_\_\_\_\_\_\_\_\_\_\_\_\_\_\_\_\_\_\_\_\_\_\_\_\_\_\_\_\_\_\_\_\_\_\_\_\_\_\_\_\_\_\_\_\_\_\_\_\_\_\_\_\_\_\_\_\_\_\_\_\_\_\_\_\_\_\_\_\_\_\_\_\_\_\_\_\_\_\_\_\_\_\_\_\_\_\_\_\_\_\_\_\_\_\_\_\_\_\_\_\_\_ \\\_\_\_\_\_\_\_\_\_\_\_\_\_\_\_\_\_\_\_\_\_\_\_\_\_\_\_\_\_\_\_\_\_\_\_\_\_\_\_\_\_\_\_\_\_\_\_\_\_\_\_\_\_\_\_\_\_\_\_\_\_\_\_\_\_\_\_\_\_\_\_\_\_\_\_\_\_\_\_\_\_\_\_\_\_\_\_\_\_\_\_\_\_\_\_\_\_\_\_\_\_\_\_\_\_\_\_\_\_\_\_\_\_\_\_\_\_\_\_\_\_\_\_\_\_\_\_ \\\_\_\_\_\_\_\_\_\_\_\_\_\_\_\_\_\_\_\_\_\_\_\_\_\_\_\_\_\_\_\_\_\_\_\_\_\_\_\_\_\_\_\_\_\_\_\_\_\_\_\_\_\_\_\_\_\_\_\_\_\_\_\_\_\_\_\_\_\_\_\_\_\_\_\_\_\_\_\_\_\_\_\_\_\_\_\_\_\_\_\_\_\_\_\_\_\_\_\_\_\_\_\_\_\_\_\_\_\_\_\_\_\_\_\_\_\_\_\_\_\_\_\_\_\_\_\_ \\\_\_\_\_\_\_\_\_\_\_\_\_\_\_\_\_\_\_\_\_\_\_\_\_\_\_\_\_\_\_\_\_\_\_\_\_\_\_\_\_\_\_\_\_\_\_\_\_\_\_\_\_\_\_\_\_\_\_\_\_\_\_\_\_\_\_\_\_\_\_\_\_\_\_\_\_\_\_\_\_\_\_\_\_\_\_\_\_\_\_\_\_\_\_\_\_\_\_\_\_\_\_\_\_\_\_\_\_\_\_\_\_\_\_\_\_\_\_\_\_\_\_\_\_\_\_\_ \\\_\_\_\_\_\_\_\_\_\_\_\_\_\_\_\_\_\_\_\_\_\_\_\_\_\_\_\_\_\_\_\_\_\_\_\_\_\_\_\_\_\_\_\_\_\_\_\_\_\_\_\_\_\_\_\_\_\_\_\_\_\_\_\_\_\_\_\_\_\_\_\_\_\_\_\_\_\_\_\_\_\_\_\_\_\_\_\_\_\_\_\_\_\_\_\_\_\_\_\_\_\_\_\_\_\_\_\_\_\_\_\_\_\_\_\_\_\_\_\_\_\_\_\_\_\_\_ \\\_\_\_\_\_\_\_\_\_\_\_\_\_\_\_\_\_\_\_\_\_\_\_\_\_\_\_\_\_\_\_\_\_\_\_\_\_\_\_\_\_\_\_\_\_\_\_\_\_\_\_\_\_\_\_\_\_\_\_\_\_\_\_\_\_\_\_\_\_\_\_\_\_\_\_\_\_\_\_\_\_\_\_\_\_\_\_\_\_\_\_\_\_\_\_\_\_\_\_\_\_\_\_\_\_\_\_\_\_\_\_\_\_\_\_\_\_\_\_\_\_\_\_\_\_\_\_ \\\_\_\_\_\_\_\_\_\_\_\_\_\_\_\_\_\_\_\_\_\_\_\_\_\_\_\_\_\_\_\_\_\_\_\_\_\_\_\_\_\_\_\_\_\_\_\_\_\_\_\_\_\_\_\_\_\_\_\_\_\_\_\_\_\_\_\_\_\_\_\_\_\_\_\_\_\_\_\_\_\_\_\_\_\_\_\_\_\_\_\_\_\_\_\_\_\_\_\_\_\_\_\_\_\_\_\_\_\_\_\_\_\_\_\_\_\_\_\_\_\_\_\_\_\_\_\_ \\\_\_\_\_\_\_\_\_\_\_\_\_\_\_\_\_\_\_\_\_\_\_\_\_\_\_\_\_\_\_\_\_\_\_\_\_\_\_\_\_\_\_\_\_\_\_\_\_\_\_\_\_\_\_\_\_\_\_\_\_\_\_\_\_\_\_\_\_\_\_\_\_\_\_\_\_\_\_\_\_\_\_\_\_\_\_\_\_\_\_\_\_\_\_\_\_\_\_\_\_\_\_\_\_\_\_\_\_\_\_\_\_\_\_\_\_\_\_\_\_\_\_\_\_\_\_\_ \\\_\_\_\_\_\_\_\_\_\_\_\_\_\_\_\_\_\_\_\_\_\_\_\_\_\_\_\_\_\_\_\_\_\_\_\_\_\_\_\_\_\_\_\_\_\_\_\_\_\_\_\_\_\_\_\_\_\_\_\_\_\_\_\_\_\_\_\_\_\_\_\_\_\_\_\_\_\_\_\_\_\_\_\_\_\_\_\_\_\_\_\_\_\_\_\_\_\_\_\_\_\_\_\_\_\_\_\_\_\_\_\_\_\_\_\_\_\_\_\_\_\_\_\_\_\_\_ \\\_\_\_\_\_\_\_\_\_\_\_\_\_\_\_\_\_\_\_\_\_\_\_\_\_\_\_\_\_\_\_\_\_\_\_\_\_\_\_\_\_\_\_\_\_\_\_\_\_\_\_\_\_\_\_\_\_\_\_\_\_\_\_\_\_\_\_\_\_\_\_\_\_\_\_\_\_\_\_\_\_\_\_\_\_\_\_\_\_\_\_\_\_\_\_\_\_\_\_\_\_\_\_\_\_\_\_\_\_\_\_\_\_\_\_\_\_\_\_\_\_\_\_\_\_\_\_ \\\_\_\_\_\_\_\_\_\_\_\_\_\_\_\_\_\_\_\_\_\_\_\_\_\_\_\_\_\_\_\_\_\_\_\_\_\_\_\_\_\_\_\_\_\_\_\_\_\_\_\_\_\_\_\_\_\_\_\_\_\_\_\_\_\_\_\_\_\_\_\_\_\_\_\_\_\_\_\_\_\_\_\_\_\_\_\_\_\_\_\_\_\_\_\_\_\_\_\_\_\_\_\_\_\_\_\_\_\_\_\_\_\_\_\_\_\_\_\_\_\_\_\_\_\_\_\_ \\\_\_\_\_\_\_\_\_\_\_\_\_\_\_\_\_\_\_\_\_\_\_\_\_\_\_\_\_\_\_\_\_\_\_\_\_\_\_\_\_\_\_\_\_\_\_\_\_\_\_\_\_\_\_\_\_\_\_\_\_\_\_\_\_\_\_\_\_\_\_\_\_\_\_\_\_\_\_\_\_\_\_\_\_\_\_\_\_\_\_\_\_\_\_\_\_\_\_\_\_\_\_\_\_\_\_\_\_\_\_\_\_\_\_\_\_\_\_\_\_\_\_\_\_\_\_\_ \\\_\_\_\_\_\_\_\_\_\_\_\_\_\_\_\_\_\_\_\_\_\_\_\_\_\_\_\_\_\_\_\_\_\_\_\_\_\_\_\_\_\_\_\_\_\_\_\_\_\_\_\_\_\_\_\_\_\_\_\_\_\_\_\_\_\_\_\_\_\_\_\_\_\_\_\_\_\_\_\_\_\_\_\_\_\_\_\_\_\_\_\_\_\_\_\_\_\_\_\_\_\_\_\_\_\_\_\_\_\_\_\_\_\_\_\_\_\_\_\_\_\_\_\_\_\_\_
\\\_\_\_\_\_\_\_\_\_\_\_\_\_\_\_\_\_\_\_\_\_\_\_\_\_\_\_\_\_\_\_\_\_\_\_\_\_\_\_\_\_\_\_\_\_\_\_\_\_\_\_\_\_\_\_\_\_\_\_\_\_\_\_\_\_\_\_\_\_\_\_\_\_\_\_\_\_\_\_\_\_\_\_\_\_\_\_\_\_\_\_\_\_\_\_\_\_\_\_\_\_\_\_\_\_\_\_\_\_\_\_\_\_\_\_\_\_\_\_\_\_\_\_\_\_\_\_
\\\_\_\_\_\_\_\_\_\_\_\_\_\_\_\_\_\_\_\_\_\_\_\_\_\_\_\_\_\_\_\_\_\_\_\_\_\_\_\_\_\_\_\_\_\_\_\_\_\_\_\_\_\_\_\_\_\_\_\_\_\_\_\_\_\_\_\_\_\_\_\_\_\_\_\_\_\_\_\_\_\_\_\_\_\_\_\_\_\_\_\_\_\_\_\_\_\_\_\_\_\_\_\_\_\_\_\_\_\_\_\_\_\_\_\_\_\_\_\_\_\_\_\_\_\_\_\_
\\\_\_\_\_\_\_\_\_\_\_\_\_\_\_\_\_\_\_\_\_\_\_\_\_\_\_\_\_\_\_\_\_\_\_\_\_\_\_\_\_\_\_\_\_\_\_\_\_\_\_\_\_\_\_\_\_\_\_\_\_\_\_\_\_\_\_\_\_\_\_\_\_\_\_\_\_\_\_\_\_\_\_\_\_\_\_\_\_\_\_\_\_\_\_\_\_\_\_\_\_\_\_\_\_\_\_\_\_\_\_\_\_\_\_\_\_\_\_\_\_\_\_\_\_\_\_\_
\\\_\_\_\_\_\_\_\_\_\_\_\_\_\_\_\_\_\_\_\_\_\_\_\_\_\_\_\_\_\_\_\_\_\_\_\_\_\_\_\_\_\_\_\_\_\_\_\_\_\_\_\_\_\_\_\_\_\_\_\_\_\_\_\_\_\_\_\_\_\_\_\_\_\_\_\_\_\_\_\_\_\_\_\_\_\_\_\_\_\_\_\_\_\_\_\_\_\_\_\_\_\_\_\_\_\_\_\_\_\_\_\_\_\_\_\_\_\_\_\_\_\_\_\_\_\_\_
\\\_\_\_\_\_\_\_\_\_\_\_\_\_\_\_\_\_\_\_\_\_\_\_\_\_\_\_\_\_\_\_\_\_\_\_\_\_\_\_\_\_\_\_\_\_\_\_\_\_\_\_\_\_\_\_\_\_\_\_\_\_\_\_\_\_\_\_\_\_\_\_\_\_\_\_\_\_\_\_\_\_\_\_\_\_\_\_\_\_\_\_\_\_\_\_\_\_\_\_\_\_\_\_\_\_\_\_\_\_\_\_\_\_\_\_\_\_\_\_\_\_\_\_\_\_\_\_
\\\_\_\_\_\_\_\_\_\_\_\_\_\_\_\_\_\_\_\_\_\_\_\_\_\_\_\_\_\_\_\_\_\_\_\_\_\_\_\_\_\_\_\_\_\_\_\_\_\_\_\_\_\_\_\_\_\_\_\_\_\_\_\_\_\_\_\_\_\_\_\_\_\_\_\_\_\_\_\_\_\_\_\_\_\_\_\_\_\_\_\_\_\_\_\_\_\_\_\_\_\_\_\_\_\_\_\_\_\_\_\_\_\_\_\_\_\_\_\_\_\_\_\_\_\_\_\_
\\\_\_\_\_\_\_\_\_\_\_\_\_\_\_\_\_\_\_\_\_\_\_\_\_\_\_\_\_\_\_\_\_\_\_\_\_\_\_\_\_\_\_\_\_\_\_\_\_\_\_\_\_\_\_\_\_\_\_\_\_\_\_\_\_\_\_\_\_\_\_\_\_\_\_\_\_\_\_\_\_\_\_\_\_\_\_\_\_\_\_\_\_\_\_\_\_\_\_\_\_\_\_\_\_\_\_\_\_\_\_\_\_\_\_\_\_\_\_\_\_\_\_\_\_\_\_\_
\\\_\_\_\_\_\_\_\_\_\_\_\_\_\_\_\_\_\_\_\_\_\_\_\_\_\_\_\_\_\_\_\_\_\_\_\_\_\_\_\_\_\_\_\_\_\_\_\_\_\_\_\_\_\_\_\_\_\_\_\_\_\_\_\_\_\_\_\_\_\_\_\_\_\_\_\_\_\_\_\_\_\_\_\_\_\_\_\_\_\_\_\_\_\_\_\_\_\_\_\_\_\_\_\_\_\_\_\_\_\_\_\_\_\_\_\_\_\_\_\_\_\_\_\_\_\_\_
\\\_\_\_\_\_\_\_\_\_\_\_\_\_\_\_\_\_\_\_\_\_\_\_\_\_\_\_\_\_\_\_\_\_\_\_\_\_\_\_\_\_\_\_\_\_\_\_\_\_\_\_\_\_\_\_\_\_\_\_\_\_\_\_\_\_\_\_\_\_\_\_\_\_\_\_\_\_\_\_\_\_\_\_\_\_\_\_\_\_\_\_\_\_\_\_\_\_\_\_\_\_\_\_\_\_\_\_\_\_\_\_\_\_\_\_\_\_\_\_\_\_\_\_\_\_\_\_
\\\_\_\_\_\_\_\_\_\_\_\_\_\_\_\_\_\_\_\_\_\_\_\_\_\_\_\_\_\_\_\_\_\_\_\_\_\_\_\_\_\_\_\_\_\_\_\_\_\_\_\_\_\_\_\_\_\_\_\_\_\_\_\_\_\_\_\_\_\_\_\_\_\_\_\_\_\_\_\_\_\_\_\_\_\_\_\_\_\_\_\_\_\_\_\_\_\_\_\_\_\_\_\_\_\_\_\_\_\_\_\_\_\_\_\_\_\_\_\_\_\_\_\_\_\_\_\_
\\\_\_\_\_\_\_\_\_\_\_\_\_\_\_\_\_\_\_\_\_\_\_\_\_\_\_\_\_\_\_\_\_\_\_\_\_\_\_\_\_\_\_\_\_\_\_\_\_\_\_\_\_\_\_\_\_\_\_\_\_\_\_\_\_\_\_\_\_\_\_\_\_\_\_\_\_\_\_\_\_\_\_\_\_\_\_\_\_\_\_\_\_\_\_\_\_\_\_\_\_\_\_\_\_\_\_\_\_\_\_\_\_\_\_\_\_\_\_\_\_\_\_\_\_\_\_\_\\
\_\_\_\_\_\_\_\_\_\_\_\_\_\_\_\_\_\_\_\_\_\_\_\_\_\_\_\_\_\_\_\_\_\_\_\_\_\_\_\_\_\_\_\_\_\_\_\_\_\_\_\_\_\_\_\_\_\_\_\_\_\_\_\_\_\_\_\_\_\_\_\_\_\_\_\_\_\_\_\_\_\_\_\_\_\_\_\_\_\_\_\_\_\_\_\_\_\_\_\_\_\_\_\_\_\_\_\_\_\_\_\_\_\_\_\_\_\_\_\_\_\_\_\_\_\_\_\\
\_\_\_\_\_\_\_\_\_\_\_\_\_\_\_\_\_\_\_\_\_\_\_\_\_\_\_\_\_\_\_\_\_\_\_\_\_\_\_\_\_\_\_\_\_\_\_\_\_\_\_\_\_\_\_\_\_\_\_\_\_\_\_\_\_\_\_\_\_\_\_\_\_\_\_\_\_\_\_\_\_\_\_\_\_\_\_\_\_\_\_\_\_\_\_\_\_\_\_\_\_\_\_\_\_\_\_\_\_\_\_\_\_\_\_\_\_\_\_\_\_\_\_\_\_\_\_\\
\_\_\_\_\_\_\_\_\_\_\_\_\_\_\_\_\_\_\_\_\_\_\_\_\_\_\_\_\_\_\_\_\_\_\_\_\_\_\_\_\_\_\_\_\_\_\_\_\_\_\_\_\_\_\_\_\_\_\_\_\_\_\_\_\_\_\_\_\_\_\_\_\_\_\_\_\_\_\_\_\_\_\_\_\_\_\_\_\_\_\_\_\_\_\_\_\_\_\_\_\_\_\_\_\_\_\_\_\_\_\_\_\_\_\_\_\_\_\_\_\_\_\_\_\_\_\_\\

\textit{\textbf{Per le eventuali irregolarità l'Ufficio si riserva ogni doverosa e opportuna azione amministrativa con successivi atti.}}


Si dà atto che durante le operazioni di verifica, svoltesi con la continua assistenza della parte, non sono stati arrecati danni né agli apparecchi oggetto di controllo né ai beni mobili e immobili, che nulla è stato asportato e che la parte non ha nulla da lamentare sull’operato dei verbalizzanti.\\
Si dà atto, altresì, che le operazioni si sono protratte per il tempo strettamente necessario allo svolgimento delle stesse, ed hanno avuto termine alle ore \_\_\_\_\_\_.\_\_\_\_\_\_ dello stesso giorno.\\
Il presente atto, che si compone di n. 6 facciate e n° \_\_\_\_\_\_ allegati, è redatto in n° 2 originali, uno per l’Ufficio dei Monopoli per le Marche dell’Agenzia delle Dogane e dei Monopoli - Sede di Pesaro, ed uno per la parte.\\
Il verbale, letto e confermato senza ulteriori osservazioni ed eccezione di sorta, viene sottoscritto dai funzionari verbalizzanti e dalla parte, a cui si rilascia copia.\\


Allegati:
\begin{itemize}[label={}]
    \item \begin{math}\square\end{math} Allegato A \begin{math}\square\end{math} Allegato B \begin{math}\square\end{math} Allegato C \begin{math}\square\end{math} Documentazione fotografica
    \item \begin{math}\square\end{math} Letture \begin{math}\square\end{math} Altro \_\_\_\_\_\_\_\_\_\_\_\_\_\_\_\_\_\_\_\_\_\_\_\_\_\_\_\_\_\_\_\_\_\_\_\_\_\_\_\_\_\_\_\_\_\_\_\_\_\_\_\_\_\_\_\_\_\_\_\_
\end{itemize}

Recapiti per l'invio di eventuale documentazione:
\begin{itemize}[label={}]
    \item EMAIL: monopoli.marche.urp@adm.gov.it
    \item PEC: monopoli.ancona@pec.adm.gov.it
\end{itemize}

\signature


\end{document}