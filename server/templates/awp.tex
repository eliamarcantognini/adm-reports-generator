\documentclass[12pt]{article}
\usepackage[a4paper, top=2cm, bottom=3cm, left=1cm, right=1cm]{geometry}
\usepackage[utf8]{inputenc}
\usepackage{graphicx}
\usepackage{caption}
\usepackage{multirow}
\usepackage{makecell}
\usepackage{tabularx}
\usepackage{amssymb}
\usepackage{parskip}
\usepackage{fancyhdr}
\usepackage{enumitem}
\usepackage{lastpage}
\newcommand\denominazioneEsercizio{$denominazioneEsercizio}
\newcommand\indirizzoEsercizio{$indirizzoEsercizio}
\newcommand\codiceEsercizio{$codiceEsercizio}
\newcommand\picfAttivita{$picfAttivita}
\newcommand\tipoAttivita{$tipoAttivita}
\newcommand\superficieAttivita{$superficieAttivita}
\newcommand\denominazioneEsercente{$denominazioneEsercente}
\newcommand\cfEsercente{$cfEsercente}
\newcommand\verbalizzanteuno{$verbalizzante1}
\newcommand\verbalizzantedue{$verbalizzante2}
\newcommand\verbalizzantetre{$verbalizzante3}
\newcommand\ordineDiAccesso{$ordineDiAccesso}
\newcommand\annoIscrizione{$annoIscrizione}
\newcommand\dataVerifica{$dataVerifica}

\newcommand\signature{%
  \par\vspace{8ex}\noindent
  \begin{tabular}[t]{@{}c@{}}
    I VERBALIZZANTI\\ \\
    \makebox[15em]{\dotfill}\\
    \\
    \makebox[15em]{\dotfill}
  \end{tabular}
  \hfill
  \begin{tabular}[t]{@{}c@{}}
    LA PARTE\\ \\
    \makebox[15em]{\dotfill}
  \end{tabular}
}

\pagestyle{fancy}
\fancyhf{}
\setlength{\headheight}{1.5cm}
\lhead{\includegraphics[height=1cm]{logo-adm.jpg}}
\rhead{\scriptsize{Segue processo verbale di operazioni compiute e constatazione}\\\scriptsize{nei confronti di \denominazioneEsercente}\\\scriptsize{redatto in data \dataVerifica}}
\cfoot{Pagina \thepage \hspace{1pt} di \pageref{LastPage}}


\begin{document}
%%%%%%%%%%%%%%%%%%%%%%%%%%%%%%%%%%%%
%%%%%%%%%%%%%%%%%%%%%%%%%%%%%%%%%%%%
%               FRONTESPIZIO
%%%%%%%%%%%%%%%%%%%%%%%%%%%%%%%%%%%%
%%%%%%%%%%%%%%%%%%%%%%%%%%%%%%%%%%%%
\thispagestyle{empty}

\begin{figure}[h]
    \centering
    \includegraphics{logo-adm.jpg}
    \caption*{Ufficio dei Monopoli per le Marche\\Sezione distaccata di Pesaro}
    \label{fig:logoadm}
\end{figure}

{\centering
    \textbf{PROCESSO VERBALE DI OPERAZIONI COMPIUTE E CONSTATAZIONE}\\in materia di apparecchi da divertimento e intrattenimento
    \par
}

\begin{tabularx}{\linewidth}{|c|l|}
   \hline
    \multirowcell{3}{DENOMINAZIONE\\ ESERCIZIO} & \\ & \denominazioneEsercizio \\ & \\
    \hline
    \multirowcell{5}{LUOGO DELLA \\VERIFICA} & \\ & Indirizzo: \indirizzoEsercizio \\ & Telefono:\\ & Codice esercizio: \codiceEsercizio \\ & \\
    \hline
    \multirowcell{5}{DATI \\ ESERCIZIO} & \\ & Attività: \tipoAttivita \\ & Superficie mq.: \superficieAttivita \\ &  P.IVA/C.F.: \picfAttivita \\ & \\
    \hline
    \multirowcell{5}{DATI \\ ESERCENTE} & \\ & Denominazione Esercente: \denominazioneEsercente \\ & Indirizzo: \indirizzoEsercizio\\ & Codice Fiscale: \cfEsercente\\ & \\
    \hline
    \multirowcell{9}{RAPPRESENTANTE\\ IN ATTI \\ \scriptsize{(persona presente al}\\ \scriptsize{momento della verifica)}} & \\ & COGNOME: \_\_\_\_\_\_\_\_\_\_\_\_\_\_\_\_\_\_\_\_\_\_\_\_\_\_\_\_\_\_\_\_ NOME: \_\_\_\_\_\_\_\_\_\_\_\_\_\_\_\_\_\_\_\_\_\_\_\_\_\_\_\_\_\_\_\_\_\_ \\ & NATO A: \_\_\_\_\_\_\_\_\_\_\_\_\_\_\_\_\_\_\_\_\_\_\_\_\_\_\_\_\_\_\_\_\_\_\_\_\_\_\_\_\_\_\_\_\_\_\_\_\_\_\_\_ IL: \_\_\_\_\_\_\_\_\_\_\_\_\_\_\_\_\_\_\_\_\_\_\_\_ \\ & RESIDENTE IN: \_\_\_\_\_\_\_\_\_\_\_\_\_\_\_\_\_\_\_\_\_\_\_\_\_\_\_\_\_\_\_\_\_\_\_\_\_\_\_\_\_\_\_\_\_\_\_\_\_\_\_\_\_\_\_\_\_\_\_\_\_\_\_\_\_\_\_\_\_\_\_\_ \\ & INDIRIZZO: \_\_\_\_\_\_\_\_\_\_\_\_\_\_\_\_\_\_\_\_\_\_\_\_\_\_\_\_\_\_\_\_\_\_\_\_\_\_\_\_\_\_\_\_\_\_\_\_\_\_\_\_\_\_\_\_\_\_\_\_\_\_\_\_\_\_\_\_\_\_\_\_\_\_\_\_\_\_ \\ & Identificato a mezzo \begin{math}\;\square\end{math} carta d'identità \begin{math}\;\square\end{math} patente di guida \begin{math}\;\square\end{math} altro \\ & rilasciato/a da \_\_\_\_\_\_\_\_\_\_\_\_\_\_\_\_\_\_\_\_\_\_\_\_\_\_\_\_\_\_\_\_\_\_\_\_\_\_\_\_\_\_\_ scadente il \_\_\_\_\_\_\_\_\_\_\_\_\_\_\_\_\_\_ \\ & in qualità di \_\_\_\_\_\_\_\_\_\_\_\_\_\_\_\_\_\_\_\_\_\_\_\_\_\_\_\_\_\_\_\_\_\_\_\_\_\_\_\_\_\_\_\_\_\_\_\_\_\_\_\_\_\_\_\_\_\_\_\_\_\_\_\_\_\_\_\_\_\_\_\_\_\_\_\_\_\_ \\ & \\
    \hline
    \multirowcell{7}{AGENZIA DELLE\\DOGANE\\E DEI MONOPOLI\\ \scriptsize{Ufficio dei Monopoli per}\\ \scriptsize{le Marche}\\ \scriptsize{Sede di Pesaro}} & \\ & Verbalizzante: \verbalizzanteuno \\ & \\ & Verbalizzante: \verbalizzantedue \\ & \\ & Verbalizzante: \verbalizzantetre  \\ & \\
    \hline
\end{tabularx}


Data verifica: \dataVerifica \\
Inizio verifica: \_\_\_\_\_\_.\_\_\_\_\_\_.
\newpage
%%%%%%%%%%%%%%%%%%%%%%%%%%%%%%%%%%%%
%%%%%%%%%%%%%%%%%%%%%%%%%%%%%%%%%%%%
%               CONTENUTO
%%%%%%%%%%%%%%%%%%%%%%%%%%%%%%%%%%%%
%%%%%%%%%%%%%%%%%%%%%%%%%%%%%%%%%%%%

In data odierna \dataVerifica , alle ore \_\_\_\_\_.\_\_\_\_\_, i sopracitati dipendenti, funzionari dell'Agenzia delle Dogane e dei Monopoli – Ufficio dei Monopoli per le Marche, Sede di Pesaro, 
\begin{itemize}
    \item ai sensi dell’art. 15, comma 8-duodecies del D.L. 1° luglio 2009, n. 78, convertito con modificazioni dalla legge 3 agosto 2009, n. 102 secondo cui “gli uffici dell’ADM, nell’adempimento dei loro compiti amministrativi e tributari, si avvalgono delle attribuzioni e dei poteri previsti dagli articoli 51 e 52 del decreto del Presidente della Repubblica 26 ottobre 1972, n. 633, e successive modificazioni, ove applicabili”;
    \item in ottemperanza dell’art. 38 della Legge 388/2000 e dell’art. 22 della Legge 289/2002, che prevede la competenza dell’Agenzia al rilascio dei nulla osta per la distribuzione e per l’esercizio di apparecchi e congegni da divertimento e intrattenimento ed al controllo sull’osservanza delle disposizioni vigenti in materia;
    \item a norma dell’art. 8 D.L. 282/2002, convertito dall’art. 1 L. 27/2003, che assegna all’Agenzia la competenza in materia di amministrazione, riscossione e contenzioso delle entrate tributarie riferite ai giochi, anche di abilità, ai concorsi pronostici, alle scommesse e agli apparecchi da divertimento e intrattenimento;
    \item a norma dell’art. 24, commi 13, 14 e 15 del D.L. 98/11, convertito con modificazioni, dalla Legge 111/11, volta alla verifica e alla rilevazione di eventuali occultamenti di base imponibile, nonché alla vigilanza sull’osservanza degli obblighi previsti dalla legge e dalle convenzioni di concessione, e stabiliti dalle norme legislative ed amministrative in materia di giochi pubblici, con o senza vincita in danaro;
    \item ai sensi dell’art. 7 D.L. 158/2012 convertito con legge 189/2012;
    \item in esecuzione dell’incarico all’uopo conferito con ordine d’accesso n. 346/RI del 04/07/2023, a norma dell’art. 13 L. 689/81 e art. 38, c.7, L. 388/d ai fini di cui all’art. 110, c. 9, TULPS, dal Direttore dell’Ufficio in intestazione e munito di apposita autorizzazione rilasciata dallo stesso Direttore a norma dell’art. 52 del D.P.R. 633/72, 
\end{itemize}

accedono nell’esercizio commerciale su indicato \begin{math} \;\square\end{math} UBICATO \begin{math}\;\square\end{math} NON UBICATO in prossimità di luoghi sensibili (art. 7 comma 9 D.L. 158/2012 convertito nella L. 189/2012)
per eseguirvi un controllo sull’esatto adempimento delle disposizioni in materia di
\begin{itemize}
    \item apparecchi da divertimento ed intrattenimento di cui all’art. 110 TULPS e normativa collegata, 
    \item apparecchi elettromeccanici di cui al comma 5 art. 14 bis D.P.R. n° 640/1972,
    \item gioco minorile di cui ai commi 24 e succ. art. 24 D.L. n° 98/2011 e comma 8 art.7 D.L. n°158/2012
\end{itemize}
Dopo aver edotto la parte degli scopi dell’accesso, nonché dei diritti e degli obblighi che vanno riconosciuti al contribuente in occasione delle verifiche di natura fiscale, e averla informata della facoltà di farsi assistere da un professionista abilitato alla difesa dinanzi agli organi di giustizia tributaria, i verbalizzanti procedono al controllo.

\newpage

\begin{center}
    \textbf{A.	AUTORIZZAZIONI AMMINISTRATIVE}
\end{center}
\begin{enumerate}
    \item \textbf{Contratto} stipulato in data \_\_\_\_\_\_/\_\_\_\_\_\_/\_\_\_\_\_\_ ai fini dell’installazione degli apparecchi da divertimento e intrattenimento di cui all’art. 110, c. 6 lett. a) b) del T.U.L.P.S 
    (\textbf{Concessionario} \_\_\_\_\_\_\_\_\_\_\_\_\_\_\_\_\_\_\_\_\_\_)
    
    \textbf{Contratto} stipulato in data \_\_\_\_\_\_/\_\_\_\_\_\_/\_\_\_\_\_\_ ai fini dell’installazione degli apparecchi da divertimento e intrattenimento di cui all’art. 110, c. 6 lett. a) b) del T.U.L.P.S
    (\textbf{Concessionario} \_\_\_\_\_\_\_\_\_\_\_\_\_\_\_\_\_\_\_\_\_\_)
    \item \textbf{Licenza} di pubblica sicurezza ai sensi dell’\textbf{art. 88} del T.U.L.P.S. rilasciata dalla Questura di \_\_\_\_\_\_\_\_\_\_\_\_ in data \_\_\_\_\_\_/\_\_\_\_\_\_/\_\_\_\_\_\_;
    \item \textbf{Licenza} ai sensi dell’\textbf{art. 86} del T.U.L.P.S. presentata al Comune di \_\_\_\_\_\_\_\_\_\_\_\_\_\_\_\_\_\_\_\_\_\_ in data \_\_\_\_\_\_/\_\_\_\_\_\_/\_\_\_\_\_\_  per “\_\_\_\_\_\_\_\_\_\_\_\_\_\_\_\_\_\_\_\_\_\_”
    
    \textbf{Licenza} ai sensi dell’\textbf{art. 86} del T.U.L.P.S. presentata al Comune di \_\_\_\_\_\_\_\_\_\_\_\_\_\_\_\_\_\_\_\_\_\_  in data \_\_\_\_\_\_/\_\_\_\_\_\_/\_\_\_\_\_\_  per “\_\_\_\_\_\_\_\_\_\_\_\_\_\_\_\_\_\_\_\_\_\_”
    \item \textbf{Tabella dei giochi proibiti}: \begin{math}\; \square \end{math} ESPOSTA \begin{math}\;\square\end{math} NON ESPOSTA 
    \item \textbf{Iscrizione all’elenco} di cui al comma 82 dell’art. 1 della legge n° 13 dicembre 2010 n° 220, sostitutivo del comma 533 dell’articolo 1 della legge 23 dicembre 2005 n° 266: \begin{math}\;\square\end{math} iscritto per l'anno \annoIscrizione \begin{math};\square\end{math}non iscritto per l'anno \annoIscrizione \_\_\_\_\_\_\_\_\_\_\_\_\_\_\_\_
    
\end{enumerate}

%%% NON DOVREBBE ESSERE B?
\begin{center}
    \textbf{A. VERIFICA PRESENZA NELL’ESERCIZIO DI MINORI DI ANNI 18}
\end{center}

\begin{enumerate}[resume]
    \item Partecipazione a giochi pubblici con vincita in denaro di soggetti minori di anni 18. Violazione di cui all’art. 24 comma 20 del D.L. n. 98/2011:
    \begin{itemize}[label={}]
        \item \begin{math}\square\end{math}Non è stata constatata la partecipazione di soggetti minori di anni 18 a giochi pubblici con vincita in denaro.
        \item \begin{math}\square\end{math}È stata constatata la partecipazione di soggetti minori di anni 18 a giochi pubblici con vincita in denaro. Il soggetto è stato  identificato nella scheda contenuta in busta sigillata e controfirmata dai verbalizzanti.    
    \end{itemize}
\end{enumerate}


\begin{center}
    \textbf{B. CONTROLLO AVVERTENZE SANITARIE SUL GIOCO D’AZZARDO PATOLOGICO (G.A.P.)}
\end{center}
\begin{enumerate}[resume]
    \item Esposizione di \textbf{materiale informativo} sul rischio di dipendenza dalla pratica dei giochi con vincite in denaro predisposto dalla \textbf{ASUR} territorialmente competente.
    \begin{math} \square\end{math} ESPOSTO \begin{math}\;\square\end{math} NON ESPOSTO
    \item Presenza su ogni \textbf{apparecchio} di cui all’art.110 \textbf{comma 6 lettera a)} del T.U.L.P.S. (cd AWP) del divieto di gioco ai minori e delle formule di avvertimento sul rischio di dipendenza dalla pratica dei giochi con vincite in denaro.
    \begin{math} \square\end{math} PRESENTI \begin{math}\;\square\end{math} NON PRESENTI
\end{enumerate}


\newpage


\begin{center}
    \textbf{C. CONTROLLO DEGLI APPARECCHI DA INTRATTENIMENTO E DIVERTIMENTO PRESENTI NELL’ESERCIZIO}
\end{center}
\begin{enumerate}[resume]
    \item \textbf{Contingentamento} apparecchi da divertimento e intrattenimento (art. 4 comma 1 del D.D. 27/07/2011). Verifica rispetto rapporto tra superficie/numero apparecchi installati e rispetto della superficie minima di ingombro di ciascun apparecchio, pari a \textbf{2 mq/cadauno }
    \\\begin{math} \square\end{math} REGOLARE \begin{math}\square\end{math} IRREGOLARE
    \item \textbf{Diversificazione} dell’offerta di gioco (Violazione dell’art. 3 del D.D. del 27 ottobre 2003 -G.U. n° 255/2003).
    \\\begin{math} \square\end{math} REGOLARE \begin{math}\square\end{math} IRREGOLARE \begin{math}\square\end{math} NON PREVISTA
\end{enumerate}

\begin{center}
    \textbf{I. APPARECCHI art. 110 commi 6 a) del TULPS}
\end{center}
\begin{enumerate}[resume]
    \item Il locale è risultato:
    \begin{math} \square\end{math} SORVEGLIATO \begin{math}\square\end{math} NON SORVEGLIATO
    \item Gli apparecchi sono installati:
    \begin{itemize}[label={}]
        \item \begin{math}\square\end{math}\textbf{ALL'INTERNO} dell'esercizio
        \item \begin{math}\square\end{math}\textbf{ALL’ESTERNO} dell’esercizio. Si intima la parte a rimuovere dall’area esterna, entro il termine perentorio di h 72:00,  e di posizionarli all’interno, nel rispetto del contingentamento previsto per l’esercizio, gli apparecchi aventi codice identificativo:\\
        \_\_\_\_\_\_\_\_\_\_\_\_\_\_\_\_\_\_\_\_\_ \_\_\_\_\_\_\_\_\_\_\_\_\_\_\_\_\_\_\_\_\_ \_\_\_\_\_\_\_\_\_\_\_\_\_\_\_\_\_\_\_\_\_ \_\_\_\_\_\_\_\_\_\_\_\_\_\_\_\_\_\_\_\_\_ \_\_\_\_\_\_\_\_\_\_\_\_\_\_\_\_\_\_\_\_\_ \\
        \textbf{\textit{In ogni caso, si procederà alla segnalazione al Comune e alla Prefettura competenti, nonché al Concessionario, per violazione della prescritta vigilanza degli apparecchi con vincita in denaro.}}
    \end{itemize}
\end{enumerate}

Sugli apparecchi di cui all’\textbf{allegato A} sono stati effettuati i seguenti controlli:

\begin{enumerate}[resume]
    \item Collegamento alla rete telematica degli apparecchi:
    \\\begin{math}\square\end{math} REGOLARE \begin{math}\square\end{math} IRREGOLARE
    \item Presenza e apposizione sugli apparecchi del nulla osta di distribuzione in originale, del nulla osta per la messa in esercizio in originale e di attestato di conformità:
    \\\begin{math} \square\end{math} PRESENTI E APPOSTI \begin{math}\square\end{math} PRESENTI E NON APPOSTI
    \\\begin{math}\square\end{math} NON VISIBILI \begin{math}\square\end{math} NON PRESENTI
    \item Gli apparecchi contrassegnati con l’asterisco (*) sono stati spenti e poi riaccesi al fine di verificare che il codice identificativo apparso a video sia lo stesso presente sui relativi nulla osta.
    \\\begin{math} \square\end{math} REGOLARE \begin{math}\square\end{math} IRREGOLARE
    \item Il proprietario degli apparecchi:
    \begin{itemize}[label={}]
        \item \begin{math}\square\end{math} non è stato contattato
        \item \begin{math}\square\end{math} è stato contattato e:
        \begin{itemize}[]
            \item \begin{math}\square\end{math} non è intervenuto \begin{math}\square\end{math} è intervenuto alle ore \_\_\_\_\_\_\_\_\_\_\_\_\_\_
            \item \begin{math}\square\end{math} non ha acconsentito \begin{math}\square\end{math} ha acconsentito \\
            all'apertura di tutti i vani dei singoli apparecchi, ai fini di verificare la regolare tenuta del libretto delle manutenzioni, l’integrità dei sigilli apposti sulla scheda di gioco, il corretto funzionamento del sistema antitamper, la presenza del doppio guscio e che il cabinet sia tra quelli presenti nella scheda esplicativa.\\
            L'esito è stato: \begin{math}\square\end{math} REGOLARE \begin{math}\square\end{math} IRREGOLARE per:\\
            \_\_\_\_\_\_\_\_\_\_\_\_\_\_\_\_\_\_\_\_\_\_\_\_\_\_\_\_\_\_\_\_\_\_\_\_\_\_\_\_\_\_\_\_\_\_\_\_\_\_\_\_\_\_\_\_\_\_\_\_\_\_\_\_\_\_\_\_\_\_\_\_\_\_\_\_\_\_\_\_\_\_\_\_\_\_\_\_\_\_\_\_\_\_\_\_\_\_\_\_\_\_\_\_\_\_\_\_\_ \\
            \_\_\_\_\_\_\_\_\_\_\_\_\_\_\_\_\_\_\_\_\_\_\_\_\_\_\_\_\_\_\_\_\_\_\_\_\_\_\_\_\_\_\_\_\_\_\_\_\_\_\_\_\_\_\_\_\_\_\_\_\_\_\_\_\_\_\_\_\_\_\_\_\_\_\_\_\_\_\_\_\_\_\_\_\_\_\_\_\_\_\_\_\_\_\_\_\_\_\_\_\_\_\_\_\_\_\_\_\_ 
        \end{itemize}
    \end{itemize}
    \item Per gli apparecchi aventi codice identificativo:\\
    \_\_\_\_\_\_\_\_\_\_\_\_\_\_\_\_\_\_\_\_\_\_\_\_\_\_\_\_\_ \_\_\_\_\_\_\_\_\_\_\_\_\_\_\_\_\_\_\_\_\_\_\_\_\_\_\_\_\_\_   \_\_\_\_\_\_\_\_\_\_\_\_\_\_\_\_\_\_\_\_\_\_\_\_\_\_\_\_\_\_ \_\_\_\_\_\_\_\_\_\_\_\_\_\_\_\_\_\_\_\_\_\_\_\_\_\_\_\_\_\_ \\
    \_\_\_\_\_\_\_\_\_\_\_\_\_\_\_\_\_\_\_\_\_\_\_\_\_\_\_\_\_ \_\_\_\_\_\_\_\_\_\_\_\_\_\_\_\_\_\_\_\_\_\_\_\_\_\_\_\_\_\_   \_\_\_\_\_\_\_\_\_\_\_\_\_\_\_\_\_\_\_\_\_\_\_\_\_\_\_\_\_\_ \_\_\_\_\_\_\_\_\_\_\_\_\_\_\_\_\_\_\_\_\_\_\_\_\_\_\_\_\_\_ 
    \begin{enumerate}
        \item si è provveduto a verificare il collegamento alla rete telematica, effettuando la lettura dei contatori di gioco:
        \begin{itemize}[label={}]
            \item \begin{math}\square\end{math} direttamente dalle schede di gioco
            \item \begin{math}\square\end{math} da collegamento esterno tramite il programma SCAAMS in uso agli operatori
            \item \begin{math}\square\end{math} contattando telefonicamente il Concessionario  \_\_\_\_\_\_\_\_\_\_\_\_\_\_\_\_\_\_\_\_\_\_\_\_\_\_\_\_\_\_
            \item \begin{math}\square\end{math} a video, entrando in contabilità “contatori generali AAMS” con l’ausilio del proprietario dell’apparecchio o del suo rappresentante
        \end{itemize}
        Tali dati, raffrontati con quelli registrati e risultanti nella Banca Dati ADM-SOGEI, sono risultati: \begin{math}\square\end{math} CONGRUENTI \begin{math}\square\end{math} NON CONGRUENTI
        \item si è provveduto ad eseguire prove di gioco inserendo \_\_\_\_\_\_\_\_\_\_\_ euro e si è riscontrato sui contatori di gioco:
    \end{enumerate}
    \begin{center}
        \begin{math}\square\end{math} l'esatto incremento \begin{math}\square\end{math} alcun incremento \begin{math}\square\end{math} incremento inesatto
    \end{center}
\end{enumerate}

\begin{center}
    \textbf{II.	APPARECCHI art. 110 commi 7 a) c) c-bis) c-ter) del TULPS }
\end{center}
Nell'esercizio sono presenti i seguenti apparecchi:

\begin{table}[h]
    \centering
    \begin{tabular}{|p{2.5cm}|p{2.5cm}|p{2.5cm}|p{2.5cm}|p{2.5cm}|p{2.5cm}|}
        \hline
        Ord & Codice \newline Apparecchio & Nulla Osta \newline Esercizio & Nulla Osta \newline Distribuzione \newline (post 2006) & Modello & Proprietario \\
        \hline
        1 \newline &  &  &  &  & \\
        \hline
        2 \newline &  &  &  &  & \\
        \hline
        3 \newline &  &  &  &  & \\
        \hline
    \end{tabular}
\end{table}

sui quali è stato verificato:

\begin{enumerate}[resume]
    \item Presenza sugli apparecchi delle targhette identificative inamovibili:
    \\\begin{math} \square\end{math} PRESENTI \begin{math}\square\end{math} NON PRESENTI
    \item Presenza e apposizione sugli apparecchi dei nulla osta:
    \\\begin{math} \square\end{math} PRESENTI \begin{math}\square\end{math} PRESENTI MA NON APPOSTI \begin{math}\square\end{math} NON PRESENTI
    \item Presenza di schermata con funzionamento a \textbf{rulli virtuali}, riproducente il gioco tipo “slot machine” di cui all’art.110 comma 6 del T.U.L.P.S: (art.3 del D.D. n.133/UDG dell’8/11/2005):
    \\\begin{math} \square\end{math} PRESENTI \begin{math}\square\end{math} NON PRESENTI
    \item presenza di \textbf{led luminosi} ad alea programmata:
    \\\begin{math} \square\end{math} PRESENTI \begin{math}\square\end{math} NON PRESENTI
\end{enumerate}



\begin{center}
    \textbf{III. APPARECCHI ELETTROMECCANICI di cui al comma 5 dell’art. 14 bis del D.P.R. n° 640/1972.}
\end{center}

Nell'esercizio sono presenti i seguenti apparecchi:
\begin{table}[h]
    \centering
    \begin{tabular}{|p{2.5cm}|p{2.5cm}|p{2.5cm}|p{2.5cm}|p{2.5cm}|p{2.5cm}|}
        \hline
        Ord & Modello & Tipo & Proprietario & Quietanza \\
        \hline
        1 \newline &  &  &  & \\
        \hline
        2 \newline &  &  &  & \\
        \hline
        3 \newline &  &  &  & \\
        \hline
        4 \newline &  &  &  & \\
        \hline
    \end{tabular}
\end{table}

\begin{table}[h]
    \centering
    \begin{tabular}{|l|l|}
        \hline
         Categoria & Descrizione \\
        \hline
         CATEGORIA \textbf{AM1} & Biliardo e apparecchi similari\\
        \hline
         CATEGORIA \textbf{AM2} & Elettrogrammofono\\
        \hline
         CATEGORIA \textbf{AM3} & Calcio Balilla, Bigliardini, Carambola e app. similari\\
        \hline
         CATEGORIA \textbf{AM4} & Flipper, gioco dei dardi (freccette)\\
        \hline
         CATEGORIA \textbf{AM5} & Congegno a vibrazione per bambini\\
        \hline
         CATEGORIA \textbf{AM7} & Gioco a gettone azionato da ruspe\\
        \hline
    \end{tabular}
\end{table}

\begin{enumerate}[resume]
    \item L’esercente, a norma del D.D. 7/8/2003 art. 3 comma 4, 
    \\\begin{math} \square\end{math} ESIBISCE \begin{math}\square\end{math} NON ESIBISCE
    \\l'originale delle quietanze di pagamento dell'Imposta sugli Intrattenimenti (I.S.I.) per l'anno \annoIscrizione
\end{enumerate}

\begin{center}
    \textbf{IV. APPARECCHI TELEMATICI PER IL GIOCO A DISTANZA}
\end{center}
\begin{enumerate}[resume]
    \item Presenza di apparecchiature che, attraverso la connessione telematica, consentono ai clienti di giocare su piattaforme di gioco messe a disposizione dai Concessionari on-line, da soggetti autorizzati all’esercizio dei giochi a distanza, ovvero da soggetti privi di qualsiasi titolo concessorio o autorizzatorio rilasciato dalle competenti autorità (art.1 comma 646/648 L. 190/2014 e art.1 comma 923 L.208/2015)
    \\\begin{math} \square\end{math} PRESENTI (Allegato C) \begin{math}\square\end{math} NON PRESENTI
    \item Presenza di apparecchiature che, attraverso la connessione telematica, offrono ai clienti “Giochi Promozionali” (art.1 comma 646/648 L. 190/2014 e art.1 comma 923 L.208/2015)
    \\\begin{math} \square\end{math} PRESENTI (Allegato C) \begin{math}\square\end{math} NON PRESENTI
\end{enumerate}


\textbf{Dalle operazioni di verifica:}
\begin{itemize}
 \renewcommand{\labelitemi}{\scriptsize\begin{math}\square\end{math}}
 \item NON SONO EMERSE IRREGOLARITÀ
 \item SONO EMERSE LE IRREGOLARITÀ DI CUI AI PUNTI \_\_\_\_\_\_\_\_\_\_\_\_\_\_\_\_\_\_\_\_\_\_\_\_\_\_\_\_\_\_\_\_\_\_\_ DEL VERBALE, per le quali si prescrive \_\_\_\_\_\_\_\_\_\_\_\_\_\_\_\_\_\_\_\_\_\_\_\_\_\_\_\_\_\_\_\_\_\_\_\_\_\_\_\_\_\_\_\_\_\_\_\_\_\_\_\_\_\_\_\_\_\_\_\_\_\_\_\_\_\_\_\_\_\_\_\_\_\_\_\_\_\_\_\_\_\_\_\_ \\
\_\_\_\_\_\_\_\_\_\_\_\_\_\_\_\_\_\_\_\_\_\_\_\_\_\_\_\_\_\_\_\_\_\_\_\_\_\_\_\_\_\_\_\_\_\_\_\_\_\_\_\_\_\_\_\_\_\_\_\_\_\_\_\_\_\_\_\_\_\_\_\_\_\_\_\_\_\_\_\_\_\_\_\_\_\_\_\_\_\_\_\_\_\_\_\_\_\_\_\_\_\_\_\_\_\_\_\_\_\_\_\_\_\_\_\_\_\_\_\_\_ \\
\_\_\_\_\_\_\_\_\_\_\_\_\_\_\_\_\_\_\_\_\_\_\_\_\_\_\_\_\_\_\_\_\_\_\_\_\_\_\_\_\_\_\_\_\_\_\_\_\_\_\_\_\_\_\_\_\_\_\_\_\_\_\_\_\_\_\_\_\_\_\_\_\_\_\_\_\_\_\_\_\_\_\_\_\_\_\_\_\_\_\_\_\_\_\_\_\_\_\_\_\_\_\_\_\_\_\_\_\_\_\_\_\_\_\_\_\_\_\_\_\_ \\
\_\_\_\_\_\_\_\_\_\_\_\_\_\_\_\_\_\_\_\_\_\_\_\_\_\_\_\_\_\_\_\_\_\_\_\_\_\_\_\_\_\_\_\_\_\_\_\_\_\_\_\_\_\_\_\_\_\_\_\_\_\_\_\_\_\_\_\_\_\_\_\_\_\_\_\_\_\_\_\_\_\_\_\_\_\_\_\_\_\_\_\_\_\_\_\_\_\_\_\_\_\_\_\_\_\_\_\_\_\_\_\_\_\_\_\_\_\_\_\_\_ \\
\_\_\_\_\_\_\_\_\_\_\_\_\_\_\_\_\_\_\_\_\_\_\_\_\_\_\_\_\_\_\_\_\_\_\_\_\_\_\_\_\_\_\_\_\_\_\_\_\_\_\_\_\_\_\_\_\_\_\_\_\_\_\_\_\_\_\_\_\_\_\_\_\_\_\_\_\_\_\_\_\_\_\_\_\_\_\_\_\_\_\_\_\_\_\_\_\_\_\_\_\_\_\_\_\_\_\_\_\_\_\_\_\_\_\_\_\_\_\_\_\_ \\
\_\_\_\_\_\_\_\_\_\_\_\_\_\_\_\_\_\_\_\_\_\_\_\_\_\_\_\_\_\_\_\_\_\_\_\_\_\_\_\_\_\_\_\_\_\_\_\_\_\_\_\_\_\_\_\_\_\_\_\_\_\_\_\_\_\_\_\_\_\_\_\_\_\_\_\_\_\_\_\_\_\_\_\_\_\_\_\_\_\_\_\_\_\_\_\_\_\_\_\_\_\_\_\_\_\_\_\_\_\_\_\_\_\_\_\_\_\_\_\_\_ \\
\_\_\_\_\_\_\_\_\_\_\_\_\_\_\_\_\_\_\_\_\_\_\_\_\_\_\_\_\_\_\_\_\_\_\_\_\_\_\_\_\_\_\_\_\_\_\_\_\_\_\_\_\_\_\_\_\_\_\_\_\_\_\_\_\_\_\_\_\_\_\_\_\_\_\_\_\_\_\_\_\_\_\_\_\_\_\_\_\_\_\_\_\_\_\_\_\_\_\_\_\_\_\_\_\_\_\_\_\_\_\_\_\_\_\_\_\_\_\_\_\_ \\
\_\_\_\_\_\_\_\_\_\_\_\_\_\_\_\_\_\_\_\_\_\_\_\_\_\_\_\_\_\_\_\_\_\_\_\_\_\_\_\_\_\_\_\_\_\_\_\_\_\_\_\_\_\_\_\_\_\_\_\_\_\_\_\_\_\_\_\_\_\_\_\_\_\_\_\_\_\_\_\_\_\_\_\_\_\_\_\_\_\_\_\_\_\_\_\_\_\_\_\_\_\_\_\_\_\_\_\_\_\_\_\_\_\_\_\_\_\_\_\_\_ \\
\_\_\_\_\_\_\_\_\_\_\_\_\_\_\_\_\_\_\_\_\_\_\_\_\_\_\_\_\_\_\_\_\_\_\_\_\_\_\_\_\_\_\_\_\_\_\_\_\_\_\_\_\_\_\_\_\_\_\_\_\_\_\_\_\_\_\_\_\_\_\_\_\_\_\_\_\_\_\_\_\_\_\_\_\_\_\_\_\_\_\_\_\_\_\_\_\_\_\_\_\_\_\_\_\_\_\_\_\_\_\_\_\_\_\_\_\_\_\_\_\_ \\
\_\_\_\_\_\_\_\_\_\_\_\_\_\_\_\_\_\_\_\_\_\_\_\_\_\_\_\_\_\_\_\_\_\_\_\_\_\_\_\_\_\_\_\_\_\_\_\_\_\_\_\_\_\_\_\_\_\_\_\_\_\_\_\_\_\_\_\_\_\_\_\_\_\_\_\_\_\_\_\_\_\_\_\_\_\_\_\_\_\_\_\_\_\_\_\_\_\_\_\_\_\_\_\_\_\_\_\_\_\_\_\_\_\_\_\_\_\_\_\_\_ \\
\item SONO EMERSE LE IRREGOLARITÀ DI CUI AL PUNTO \textbf{22)} DEL VERBALE (SEZ. APPARECCHI ELETTROMECCANICI), per le quali, \textbf{con separato atto}, l’Ufficio procederà, nei confronti del \textbf{proprietario}, a richiedere il \textbf{pagamento dell’IMPOSTA SUGLI INTRATTENUMENTI (I.S.I.)} 
\item SONO EMERSE LE IRREGOLARITÀ DI CUI AI PUNTI \_\_\_\_\_\_\_\_\_\_\_\_\_\_\_\_\_\_\_\_\_\_\_\_\_\_\_\_\_\_\_\_\_\_\_ DEL VERBALE, per le quali, con separato atto, si procederà nei termini di legge alla contestazione delle sanzioni amministrative pecuniarie e a tutto quanto previsto dalle disposizioni di legge nei confronti di tutti i soggetti interessati.\\
\_\_\_\_\_\_\_\_\_\_\_\_\_\_\_\_\_\_\_\_\_\_\_\_\_\_\_\_\_\_\_\_\_\_\_\_\_\_\_\_\_\_\_\_\_\_\_\_\_\_\_\_\_\_\_\_\_\_\_\_\_\_\_\_\_\_\_\_\_\_\_\_\_\_\_\_\_\_\_\_\_\_\_\_\_\_\_\_\_\_\_\_\_\_\_\_\_\_\_\_\_\_\_\_\_\_\_\_\_\_\_\_\_\_\_\_\_\_\_\_\_ \\
\_\_\_\_\_\_\_\_\_\_\_\_\_\_\_\_\_\_\_\_\_\_\_\_\_\_\_\_\_\_\_\_\_\_\_\_\_\_\_\_\_\_\_\_\_\_\_\_\_\_\_\_\_\_\_\_\_\_\_\_\_\_\_\_\_\_\_\_\_\_\_\_\_\_\_\_\_\_\_\_\_\_\_\_\_\_\_\_\_\_\_\_\_\_\_\_\_\_\_\_\_\_\_\_\_\_\_\_\_\_\_\_\_\_\_\_\_\_\_\_\_ \\
\_\_\_\_\_\_\_\_\_\_\_\_\_\_\_\_\_\_\_\_\_\_\_\_\_\_\_\_\_\_\_\_\_\_\_\_\_\_\_\_\_\_\_\_\_\_\_\_\_\_\_\_\_\_\_\_\_\_\_\_\_\_\_\_\_\_\_\_\_\_\_\_\_\_\_\_\_\_\_\_\_\_\_\_\_\_\_\_\_\_\_\_\_\_\_\_\_\_\_\_\_\_\_\_\_\_\_\_\_\_\_\_\_\_\_\_\_\_\_\_\_ \\
\_\_\_\_\_\_\_\_\_\_\_\_\_\_\_\_\_\_\_\_\_\_\_\_\_\_\_\_\_\_\_\_\_\_\_\_\_\_\_\_\_\_\_\_\_\_\_\_\_\_\_\_\_\_\_\_\_\_\_\_\_\_\_\_\_\_\_\_\_\_\_\_\_\_\_\_\_\_\_\_\_\_\_\_\_\_\_\_\_\_\_\_\_\_\_\_\_\_\_\_\_\_\_\_\_\_\_\_\_\_\_\_\_\_\_\_\_\_\_\_\_ \\
\_\_\_\_\_\_\_\_\_\_\_\_\_\_\_\_\_\_\_\_\_\_\_\_\_\_\_\_\_\_\_\_\_\_\_\_\_\_\_\_\_\_\_\_\_\_\_\_\_\_\_\_\_\_\_\_\_\_\_\_\_\_\_\_\_\_\_\_\_\_\_\_\_\_\_\_\_\_\_\_\_\_\_\_\_\_\_\_\_\_\_\_\_\_\_\_\_\_\_\_\_\_\_\_\_\_\_\_\_\_\_\_\_\_\_\_\_\_\_\_\_ \\
\_\_\_\_\_\_\_\_\_\_\_\_\_\_\_\_\_\_\_\_\_\_\_\_\_\_\_\_\_\_\_\_\_\_\_\_\_\_\_\_\_\_\_\_\_\_\_\_\_\_\_\_\_\_\_\_\_\_\_\_\_\_\_\_\_\_\_\_\_\_\_\_\_\_\_\_\_\_\_\_\_\_\_\_\_\_\_\_\_\_\_\_\_\_\_\_\_\_\_\_\_\_\_\_\_\_\_\_\_\_\_\_\_\_\_\_\_\_\_\_\_ \\
\_\_\_\_\_\_\_\_\_\_\_\_\_\_\_\_\_\_\_\_\_\_\_\_\_\_\_\_\_\_\_\_\_\_\_\_\_\_\_\_\_\_\_\_\_\_\_\_\_\_\_\_\_\_\_\_\_\_\_\_\_\_\_\_\_\_\_\_\_\_\_\_\_\_\_\_\_\_\_\_\_\_\_\_\_\_\_\_\_\_\_\_\_\_\_\_\_\_\_\_\_\_\_\_\_\_\_\_\_\_\_\_\_\_\_\_\_\_\_\_\_ \\
\end{itemize}

\begin{center}
    \textbf{Dichiarazioni della parte}
\end{center}

\_\_\_\_\_\_\_\_\_\_\_\_\_\_\_\_\_\_\_\_\_\_\_\_\_\_\_\_\_\_\_\_\_\_\_\_\_\_\_\_\_\_\_\_\_\_\_\_\_\_\_\_\_\_\_\_\_\_\_\_\_\_\_\_\_\_\_\_\_\_\_\_\_\_\_\_\_\_\_\_\_\_\_\_\_\_\_\_\_\_\_\_\_\_\_\_\_\_\_\_\_\_\_\_\_\_\_\_\_\_\_\_\_\_\_\_\_\_\_\_\_\_\_\_\_\_\_ \\
\_\_\_\_\_\_\_\_\_\_\_\_\_\_\_\_\_\_\_\_\_\_\_\_\_\_\_\_\_\_\_\_\_\_\_\_\_\_\_\_\_\_\_\_\_\_\_\_\_\_\_\_\_\_\_\_\_\_\_\_\_\_\_\_\_\_\_\_\_\_\_\_\_\_\_\_\_\_\_\_\_\_\_\_\_\_\_\_\_\_\_\_\_\_\_\_\_\_\_\_\_\_\_\_\_\_\_\_\_\_\_\_\_\_\_\_\_\_\_\_\_\_\_\_\_\_\_ \\\_\_\_\_\_\_\_\_\_\_\_\_\_\_\_\_\_\_\_\_\_\_\_\_\_\_\_\_\_\_\_\_\_\_\_\_\_\_\_\_\_\_\_\_\_\_\_\_\_\_\_\_\_\_\_\_\_\_\_\_\_\_\_\_\_\_\_\_\_\_\_\_\_\_\_\_\_\_\_\_\_\_\_\_\_\_\_\_\_\_\_\_\_\_\_\_\_\_\_\_\_\_\_\_\_\_\_\_\_\_\_\_\_\_\_\_\_\_\_\_\_\_\_\_\_\_\_ \\\_\_\_\_\_\_\_\_\_\_\_\_\_\_\_\_\_\_\_\_\_\_\_\_\_\_\_\_\_\_\_\_\_\_\_\_\_\_\_\_\_\_\_\_\_\_\_\_\_\_\_\_\_\_\_\_\_\_\_\_\_\_\_\_\_\_\_\_\_\_\_\_\_\_\_\_\_\_\_\_\_\_\_\_\_\_\_\_\_\_\_\_\_\_\_\_\_\_\_\_\_\_\_\_\_\_\_\_\_\_\_\_\_\_\_\_\_\_\_\_\_\_\_\_\_\_\_ \\\_\_\_\_\_\_\_\_\_\_\_\_\_\_\_\_\_\_\_\_\_\_\_\_\_\_\_\_\_\_\_\_\_\_\_\_\_\_\_\_\_\_\_\_\_\_\_\_\_\_\_\_\_\_\_\_\_\_\_\_\_\_\_\_\_\_\_\_\_\_\_\_\_\_\_\_\_\_\_\_\_\_\_\_\_\_\_\_\_\_\_\_\_\_\_\_\_\_\_\_\_\_\_\_\_\_\_\_\_\_\_\_\_\_\_\_\_\_\_\_\_\_\_\_\_\_\_ \\\_\_\_\_\_\_\_\_\_\_\_\_\_\_\_\_\_\_\_\_\_\_\_\_\_\_\_\_\_\_\_\_\_\_\_\_\_\_\_\_\_\_\_\_\_\_\_\_\_\_\_\_\_\_\_\_\_\_\_\_\_\_\_\_\_\_\_\_\_\_\_\_\_\_\_\_\_\_\_\_\_\_\_\_\_\_\_\_\_\_\_\_\_\_\_\_\_\_\_\_\_\_\_\_\_\_\_\_\_\_\_\_\_\_\_\_\_\_\_\_\_\_\_\_\_\_\_ \\\_\_\_\_\_\_\_\_\_\_\_\_\_\_\_\_\_\_\_\_\_\_\_\_\_\_\_\_\_\_\_\_\_\_\_\_\_\_\_\_\_\_\_\_\_\_\_\_\_\_\_\_\_\_\_\_\_\_\_\_\_\_\_\_\_\_\_\_\_\_\_\_\_\_\_\_\_\_\_\_\_\_\_\_\_\_\_\_\_\_\_\_\_\_\_\_\_\_\_\_\_\_\_\_\_\_\_\_\_\_\_\_\_\_\_\_\_\_\_\_\_\_\_\_\_\_\_ \\\_\_\_\_\_\_\_\_\_\_\_\_\_\_\_\_\_\_\_\_\_\_\_\_\_\_\_\_\_\_\_\_\_\_\_\_\_\_\_\_\_\_\_\_\_\_\_\_\_\_\_\_\_\_\_\_\_\_\_\_\_\_\_\_\_\_\_\_\_\_\_\_\_\_\_\_\_\_\_\_\_\_\_\_\_\_\_\_\_\_\_\_\_\_\_\_\_\_\_\_\_\_\_\_\_\_\_\_\_\_\_\_\_\_\_\_\_\_\_\_\_\_\_\_\_\_\_ \\\_\_\_\_\_\_\_\_\_\_\_\_\_\_\_\_\_\_\_\_\_\_\_\_\_\_\_\_\_\_\_\_\_\_\_\_\_\_\_\_\_\_\_\_\_\_\_\_\_\_\_\_\_\_\_\_\_\_\_\_\_\_\_\_\_\_\_\_\_\_\_\_\_\_\_\_\_\_\_\_\_\_\_\_\_\_\_\_\_\_\_\_\_\_\_\_\_\_\_\_\_\_\_\_\_\_\_\_\_\_\_\_\_\_\_\_\_\_\_\_\_\_\_\_\_\_\_ \\\_\_\_\_\_\_\_\_\_\_\_\_\_\_\_\_\_\_\_\_\_\_\_\_\_\_\_\_\_\_\_\_\_\_\_\_\_\_\_\_\_\_\_\_\_\_\_\_\_\_\_\_\_\_\_\_\_\_\_\_\_\_\_\_\_\_\_\_\_\_\_\_\_\_\_\_\_\_\_\_\_\_\_\_\_\_\_\_\_\_\_\_\_\_\_\_\_\_\_\_\_\_\_\_\_\_\_\_\_\_\_\_\_\_\_\_\_\_\_\_\_\_\_\_\_\_\_ 

\begin{center}
    \textbf{Dichiarazioni dei verbalizzanti}
\end{center}

\_\_\_\_\_\_\_\_\_\_\_\_\_\_\_\_\_\_\_\_\_\_\_\_\_\_\_\_\_\_\_\_\_\_\_\_\_\_\_\_\_\_\_\_\_\_\_\_\_\_\_\_\_\_\_\_\_\_\_\_\_\_\_\_\_\_\_\_\_\_\_\_\_\_\_\_\_\_\_\_\_\_\_\_\_\_\_\_\_\_\_\_\_\_\_\_\_\_\_\_\_\_\_\_\_\_\_\_\_\_\_\_\_\_\_\_\_\_\_\_\_\_\_\_\_\_\_ \\
\_\_\_\_\_\_\_\_\_\_\_\_\_\_\_\_\_\_\_\_\_\_\_\_\_\_\_\_\_\_\_\_\_\_\_\_\_\_\_\_\_\_\_\_\_\_\_\_\_\_\_\_\_\_\_\_\_\_\_\_\_\_\_\_\_\_\_\_\_\_\_\_\_\_\_\_\_\_\_\_\_\_\_\_\_\_\_\_\_\_\_\_\_\_\_\_\_\_\_\_\_\_\_\_\_\_\_\_\_\_\_\_\_\_\_\_\_\_\_\_\_\_\_\_\_\_\_ \\\_\_\_\_\_\_\_\_\_\_\_\_\_\_\_\_\_\_\_\_\_\_\_\_\_\_\_\_\_\_\_\_\_\_\_\_\_\_\_\_\_\_\_\_\_\_\_\_\_\_\_\_\_\_\_\_\_\_\_\_\_\_\_\_\_\_\_\_\_\_\_\_\_\_\_\_\_\_\_\_\_\_\_\_\_\_\_\_\_\_\_\_\_\_\_\_\_\_\_\_\_\_\_\_\_\_\_\_\_\_\_\_\_\_\_\_\_\_\_\_\_\_\_\_\_\_\_ \\\_\_\_\_\_\_\_\_\_\_\_\_\_\_\_\_\_\_\_\_\_\_\_\_\_\_\_\_\_\_\_\_\_\_\_\_\_\_\_\_\_\_\_\_\_\_\_\_\_\_\_\_\_\_\_\_\_\_\_\_\_\_\_\_\_\_\_\_\_\_\_\_\_\_\_\_\_\_\_\_\_\_\_\_\_\_\_\_\_\_\_\_\_\_\_\_\_\_\_\_\_\_\_\_\_\_\_\_\_\_\_\_\_\_\_\_\_\_\_\_\_\_\_\_\_\_\_ \\\_\_\_\_\_\_\_\_\_\_\_\_\_\_\_\_\_\_\_\_\_\_\_\_\_\_\_\_\_\_\_\_\_\_\_\_\_\_\_\_\_\_\_\_\_\_\_\_\_\_\_\_\_\_\_\_\_\_\_\_\_\_\_\_\_\_\_\_\_\_\_\_\_\_\_\_\_\_\_\_\_\_\_\_\_\_\_\_\_\_\_\_\_\_\_\_\_\_\_\_\_\_\_\_\_\_\_\_\_\_\_\_\_\_\_\_\_\_\_\_\_\_\_\_\_\_\_ \\\_\_\_\_\_\_\_\_\_\_\_\_\_\_\_\_\_\_\_\_\_\_\_\_\_\_\_\_\_\_\_\_\_\_\_\_\_\_\_\_\_\_\_\_\_\_\_\_\_\_\_\_\_\_\_\_\_\_\_\_\_\_\_\_\_\_\_\_\_\_\_\_\_\_\_\_\_\_\_\_\_\_\_\_\_\_\_\_\_\_\_\_\_\_\_\_\_\_\_\_\_\_\_\_\_\_\_\_\_\_\_\_\_\_\_\_\_\_\_\_\_\_\_\_\_\_\_ \\\_\_\_\_\_\_\_\_\_\_\_\_\_\_\_\_\_\_\_\_\_\_\_\_\_\_\_\_\_\_\_\_\_\_\_\_\_\_\_\_\_\_\_\_\_\_\_\_\_\_\_\_\_\_\_\_\_\_\_\_\_\_\_\_\_\_\_\_\_\_\_\_\_\_\_\_\_\_\_\_\_\_\_\_\_\_\_\_\_\_\_\_\_\_\_\_\_\_\_\_\_\_\_\_\_\_\_\_\_\_\_\_\_\_\_\_\_\_\_\_\_\_\_\_\_\_\_ \\\_\_\_\_\_\_\_\_\_\_\_\_\_\_\_\_\_\_\_\_\_\_\_\_\_\_\_\_\_\_\_\_\_\_\_\_\_\_\_\_\_\_\_\_\_\_\_\_\_\_\_\_\_\_\_\_\_\_\_\_\_\_\_\_\_\_\_\_\_\_\_\_\_\_\_\_\_\_\_\_\_\_\_\_\_\_\_\_\_\_\_\_\_\_\_\_\_\_\_\_\_\_\_\_\_\_\_\_\_\_\_\_\_\_\_\_\_\_\_\_\_\_\_\_\_\_\_ \\\_\_\_\_\_\_\_\_\_\_\_\_\_\_\_\_\_\_\_\_\_\_\_\_\_\_\_\_\_\_\_\_\_\_\_\_\_\_\_\_\_\_\_\_\_\_\_\_\_\_\_\_\_\_\_\_\_\_\_\_\_\_\_\_\_\_\_\_\_\_\_\_\_\_\_\_\_\_\_\_\_\_\_\_\_\_\_\_\_\_\_\_\_\_\_\_\_\_\_\_\_\_\_\_\_\_\_\_\_\_\_\_\_\_\_\_\_\_\_\_\_\_\_\_\_\_\_ \\\_\_\_\_\_\_\_\_\_\_\_\_\_\_\_\_\_\_\_\_\_\_\_\_\_\_\_\_\_\_\_\_\_\_\_\_\_\_\_\_\_\_\_\_\_\_\_\_\_\_\_\_\_\_\_\_\_\_\_\_\_\_\_\_\_\_\_\_\_\_\_\_\_\_\_\_\_\_\_\_\_\_\_\_\_\_\_\_\_\_\_\_\_\_\_\_\_\_\_\_\_\_\_\_\_\_\_\_\_\_\_\_\_\_\_\_\_\_\_\_\_\_\_\_\_\_\_ \\\_\_\_\_\_\_\_\_\_\_\_\_\_\_\_\_\_\_\_\_\_\_\_\_\_\_\_\_\_\_\_\_\_\_\_\_\_\_\_\_\_\_\_\_\_\_\_\_\_\_\_\_\_\_\_\_\_\_\_\_\_\_\_\_\_\_\_\_\_\_\_\_\_\_\_\_\_\_\_\_\_\_\_\_\_\_\_\_\_\_\_\_\_\_\_\_\_\_\_\_\_\_\_\_\_\_\_\_\_\_\_\_\_\_\_\_\_\_\_\_\_\_\_\_\_\_\_ \\\_\_\_\_\_\_\_\_\_\_\_\_\_\_\_\_\_\_\_\_\_\_\_\_\_\_\_\_\_\_\_\_\_\_\_\_\_\_\_\_\_\_\_\_\_\_\_\_\_\_\_\_\_\_\_\_\_\_\_\_\_\_\_\_\_\_\_\_\_\_\_\_\_\_\_\_\_\_\_\_\_\_\_\_\_\_\_\_\_\_\_\_\_\_\_\_\_\_\_\_\_\_\_\_\_\_\_\_\_\_\_\_\_\_\_\_\_\_\_\_\_\_\_\_\_\_\_ \\\_\_\_\_\_\_\_\_\_\_\_\_\_\_\_\_\_\_\_\_\_\_\_\_\_\_\_\_\_\_\_\_\_\_\_\_\_\_\_\_\_\_\_\_\_\_\_\_\_\_\_\_\_\_\_\_\_\_\_\_\_\_\_\_\_\_\_\_\_\_\_\_\_\_\_\_\_\_\_\_\_\_\_\_\_\_\_\_\_\_\_\_\_\_\_\_\_\_\_\_\_\_\_\_\_\_\_\_\_\_\_\_\_\_\_\_\_\_\_\_\_\_\_\_\_\_\_ \\\_\_\_\_\_\_\_\_\_\_\_\_\_\_\_\_\_\_\_\_\_\_\_\_\_\_\_\_\_\_\_\_\_\_\_\_\_\_\_\_\_\_\_\_\_\_\_\_\_\_\_\_\_\_\_\_\_\_\_\_\_\_\_\_\_\_\_\_\_\_\_\_\_\_\_\_\_\_\_\_\_\_\_\_\_\_\_\_\_\_\_\_\_\_\_\_\_\_\_\_\_\_\_\_\_\_\_\_\_\_\_\_\_\_\_\_\_\_\_\_\_\_\_\_\_\_\_ \\\_\_\_\_\_\_\_\_\_\_\_\_\_\_\_\_\_\_\_\_\_\_\_\_\_\_\_\_\_\_\_\_\_\_\_\_\_\_\_\_\_\_\_\_\_\_\_\_\_\_\_\_\_\_\_\_\_\_\_\_\_\_\_\_\_\_\_\_\_\_\_\_\_\_\_\_\_\_\_\_\_\_\_\_\_\_\_\_\_\_\_\_\_\_\_\_\_\_\_\_\_\_\_\_\_\_\_\_\_\_\_\_\_\_\_\_\_\_\_\_\_\_\_\_\_\_\_ \\

Si dà atto che durante le operazioni di verifica, svoltesi con la continua assistenza della parte, non sono stati arrecati danni né agli apparecchi oggetto di controllo né ai beni mobili e immobili, che nulla è stato asportato e che la parte non ha nulla da lamentare sull’operato dei verbalizzanti.\\
Si dà atto, altresì, che le operazioni si sono protratte per il tempo strettamente necessario allo svolgimento delle stesse, ed hanno avuto termine alle ore \_\_\_\_\_\_.\_\_\_\_\_\_ dello stesso giorno.\\
Il presente atto, che si compone di n° 8 facciate e n° \_\_\_\_\_\_ allegati, è redatto in n° 2 originali, uno per l’Ufficio dei Monopoli per le Marche dell’Agenzia delle Dogane e dei Monopoli - Sede di Pesaro, ed uno per la parte.\\
Il verbale, letto e confermato senza ulteriori osservazione ed eccezione di sorta, viene sottoscritto dai funzionari verbalizzanti e dalla parte, a cui si rilascia copia.

Allegati: 
\begin{itemize}[label={}]
    \item \begin{math}\square\end{math} Allegato A \begin{math}\square\end{math} Allegato B \begin{math}\square\end{math} Allegato C \begin{math}\square\end{math} Documentazione fotografica
    \item \begin{math}\square\end{math} Letture \begin{math}\square\end{math} Altro \_\_\_\_\_\_\_\_\_\_\_\_\_\_\_\_\_\_\_\_\_\_\_\_\_\_\_\_\_\_\_\_\_\_\_\_\_\_\_\_\_\_\_\_\_\_\_\_\_\_\_\_\_\_\_\_\_\_\_\_
\end{itemize}

Recapiti per l'invio di eventuale documentazione:
\begin{itemize}[label={}]
    \item EMAIL: monopoli.marche.urp@adm.gov.it
    \item PEC: monopoli.ancona@pec.adm.gov.it
\end{itemize}

\signature


\end{document}