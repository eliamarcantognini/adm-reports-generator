\documentclass[12pt]{article}
\usepackage[a4paper, top=2cm, bottom=3cm, left=1cm, right=1cm]{geometry}
\usepackage[utf8]{inputenc}
\usepackage{graphicx}
\usepackage{caption}
\usepackage{multirow}
\usepackage{multicol}
\usepackage{makecell}
\usepackage{tabularx}
\usepackage{amssymb}
\usepackage{parskip}
\usepackage{fancyhdr}
\usepackage{enumitem}
\usepackage{lastpage}

\newcolumntype{Y}{>{\centering\arraybackslash}X}

\newcommand\denominazioneEsercizio{$denominazioneEsercizio}
\newcommand\indirizzoEsercizio{$indirizzoEsercizio}
\newcommand\picfAttivita{$picfAttivita}
\newcommand\denominazioneEsercente{$denominazioneEsercente}
\newcommand\verbalizzanteuno{$verbalizzante1}
\newcommand\verbalizzantedue{$verbalizzante2}
\newcommand\verbalizzantetre{$verbalizzante3}
\newcommand\ordineDiAccesso{$ordineDiAccesso}
\newcommand\dataVerifica{$dataVerifica}
\newcommand\nRivendita{$nRivendita}
\newcommand\localitaRivendita{$localitaRivendita}
\newcommand\nPatentino{$nPatentino}
\newcommand\localitaPatentino{$localitaPatentino}

\newcommand\signature{%
  \par\vspace{8ex}\noindent
  \begin{tabular}[t]{@{}c@{}}
    I VERBALIZZANTI\\ \\
    \makebox[15em]{\dotfill}\\
    \\
    \makebox[15em]{\dotfill}\\
    \\
    \makebox[15em]{\dotfill}
  \end{tabular}
  \hfill
  \begin{tabular}[t]{@{}c@{}}
    LA PARTE\\ \\
    \makebox[15em]{\dotfill}
  \end{tabular}
}

\pagestyle{fancy}
\fancyhf{}
\setlength{\headheight}{1.5cm}
\lhead{\includegraphics[height=1cm]{logo-adm.jpg}}
% DA CAMBIARE
\rhead{\scriptsize{Segue processo verbale di accesso e operazioni compiute}\\\scriptsize{nei confronti di \denominazioneEsercente}\\\scriptsize{redatto in data \dataVerifica}}
\cfoot{Pagina \thepage \hspace{1pt} di \pageref{LastPage}}


\begin{document}
%%%%%%%%%%%%%%%%%%%%%%%%%%%%%%%%%%%%
%%%%%%%%%%%%%%%%%%%%%%%%%%%%%%%%%%%%
%               FRONTESPIZIO
%%%%%%%%%%%%%%%%%%%%%%%%%%%%%%%%%%%%
%%%%%%%%%%%%%%%%%%%%%%%%%%%%%%%%%%%%
\thispagestyle{empty}

\begin{figure}[h]
    \centering
    \includegraphics{logo-adm.jpg}
    \caption*{Ufficio dei Monopoli per le Marche\\Sezione distaccata di Pesaro}
    \label{fig:logoadm}
\end{figure}

{\centering
    \textbf{PROCESSO VERBALE DI CONSTATAZIONE - ACCERTAMENTO}
    \par
}

\begin{tabularx}{\linewidth}{|c|X|}
   \hline
    \multirowcell{9}{DATI \\ ESERCIZIO} & \\ & Denominazione: \denominazioneEsercizio \\ & Indirizzo: \indirizzoEsercizio \\ &  P.IVA/C.F.: \picfAttivita \\ & Titolare: \denominazioneEsercente \\& Telefono: \_\_\_\_\_\_\_\_\_\_\_\_\_\_\_\_\_\_\_\_\_\_\_\_\_\_\_\_\_\_\_\_\_\_\_\_\_\_\_\_\_\_\_\_\_\_\_\_\_\_\_\_\_\_\_\_\_\_\_\_\_\_\_\_\_\_\_\_\_\_\_\_\_\_\_\_\_\_\_\_\_\_ \\ & Orari: \_\_\_\_\_\_\_\_\_\_\_\_\_\_\_\_\_\_\_\_\_\_\_\_\_\_\_\_\_\_\_\_\_\_\_\_\_\_\_\_\_\_\_\_\_\_\_\_\_\_\_\_\_\_\_\_\_\_\_\_\_\_\_\_\_\_\_\_\_\_\_\_\_\_\_\_\_\_\_\_\_\_\_\_\_\_ \\ & Rivendita di aggregazione n. \nRivendita in \localitaRivendita \\ & Patentino n. \nPatentino in \localitaPatentino \\ & \\
    \hline
    \multirowcell{5}{OGGETTO\\DELLA\\VERIFICA} & \\ & VERIFICA PATENTINO \\ & N. \nPatentino in \localitaPatentino  \\ & PER LA VENDITA DI GENERI DI MONOPOLIO \\ & \\
    \hline
    \multirowcell{10}{ANAGRAFICA\\ DELLA \\ PARTE} & \\ & COGNOME: \_\_\_\_\_\_\_\_\_\_\_\_\_\_\_\_\_\_\_\_\_\_\_\_\_\_\_\_\_\_\_\_ NOME: \_\_\_\_\_\_\_\_\_\_\_\_\_\_\_\_\_\_\_\_\_\_\_\_\_\_\_\_\_\_\_\_\_\_ \\ & NATO A: \_\_\_\_\_\_\_\_\_\_\_\_\_\_\_\_\_\_\_\_\_\_\_\_\_\_\_\_\_\_\_\_\_\_\_\_\_\_\_\_\_\_\_\_\_\_\_\_\_\_\_\_ IL: \_\_\_\_\_\_\_\_\_\_\_\_\_\_\_\_\_\_\_\_\_\_\_\_ \\ & RESIDENTE IN: \_\_\_\_\_\_\_\_\_\_\_\_\_\_\_\_\_\_\_\_\_\_\_\_\_\_\_\_\_\_\_\_\_\_\_\_\_\_\_\_\_\_\_\_\_\_\_\_\_\_\_\_\_\_\_\_\_\_\_\_\_\_\_\_\_\_\_\_\_\_\_\_ \\ & INDIRIZZO: \_\_\_\_\_\_\_\_\_\_\_\_\_\_\_\_\_\_\_\_\_\_\_\_\_\_\_\_\_\_\_\_\_\_\_\_\_\_\_\_\_\_\_\_\_\_\_\_\_\_\_\_\_\_\_\_\_\_\_\_\_\_\_\_\_\_\_\_\_\_\_\_\_\_\_\_\_\_ \\ & Identificato a mezzo \begin{math}\;\square\end{math} carta d'identità \begin{math}\;\square\end{math} patente di guida \begin{math}\;\square\end{math} altro \\ & rilasciato/a da \_\_\_\_\_\_\_\_\_\_\_\_\_\_\_\_\_\_\_\_\_\_\_\_\_\_\_\_\_\_\_\_\_\_\_\_\_\_\_\_\_\_\_ scadente il \_\_\_\_\_\_\_\_\_\_\_\_\_\_\_\_\_\_ \\ & IN QUALITÀ DI: \\ &\begin{math}\square\end{math} titolare/legale rappresentante  \begin{math}\square\end{math} dipendente \begin{math}\square\end{math} Altro\_\_\_\_\_\_\_\_\_\_\_\_\_\_\_\_\_\_\_\_\_\_\_\_\_\_\_\_\_ \\ & \\
    \hline
    \multirowcell{7}{AGENZIA DELLE\\DOGANE\\E DEI MONOPOLI\\ \scriptsize{Ufficio dei Monopoli per}\\ \scriptsize{le Marche}\\ \scriptsize{Sede di Pesaro}} & \\ & Verbalizzante: \verbalizzanteuno \\ & \\ & Verbalizzante: \verbalizzantedue \\ & \\ & Verbalizzante: \verbalizzantetre  \\ & \\
    \hline
\end{tabularx}


Data verifica: \dataVerifica \\
Inizio verifica: \_\_\_\_\_\_.\_\_\_\_\_\_.
\newpage
%%%%%%%%%%%%%%%%%%%%%%%%%%%%%%%%%%%%
%%%%%%%%%%%%%%%%%%%%%%%%%%%%%%%%%%%%
%               CONTENUTO
%%%%%%%%%%%%%%%%%%%%%%%%%%%%%%%%%%%%
%%%%%%%%%%%%%%%%%%%%%%%%%%%%%%%%%%%%


In data odierna \dataVerifica, alle ore \_\_\_\_\_.\_\_\_\_\_, i sopracitati dipendenti, funzionari dell'Agenzia delle Dogane e dei Monopoli – Ufficio dei Monopoli per le Marche, Sede di Pesaro, ai sensi delle disposizioni:
\begin{itemize}
    \item Legge 1293/57;
    \item Decreto del Presidente della Repubblica 1074/58;
    \item Decreto Legislativo 504/95;
    \item Decreto Ministeriale 38/2013;
    \item Decreto Legilastivo 6/2016;
\end{itemize}
hanno effettuato, in esecuzione dell’ordine d’accesso \ordineDiAccesso, all’uopo conferito dal Dirigente dell’Ufficio in intestazione e ai sensi dell’art.18 del D.Lgs. 504/95 (Testo Unico delle Accise), una verifica presso l’esercizio \_\_\_\_\_\_\_\_\_\_\_\_\_\_\_\_\_\_ sito in \indirizzoEsercizio, di cui è titolare \denominazioneEsercente al fine di verificare la regolarità della vendita dei prodotti da fumo, così come stabilito dalla legge 22 dicembre 1957 n. 1293, dal relativo regolamento di attuazione e dal Capitolato d’oneri, nonché l’osservanza del divieto di vendita dei tabacchi ai minori di anni diciotto.

Dopo aver dimostrato la propria identità mediante esibizione delle tessere di servizio i funzionari ADM hanno reso edotta la parte sullo scopo dell’accesso e della facoltà di farsi assistere da un professionista abilitato alla difesa dinanzi agli organi di giustizia tributaria, nonché dei diritti e degli obblighi che vanno riconosciuti al contribuente in occasione delle verifiche ai sensi dello Statuto del Contribuente (L. n. 212/2000).\\
La parte dichiara di essere stata informata delle ragioni che hanno determinato la verifica, dell’oggetto che la riguarda, nonché, in applicazione della legge n. 212/2000 (Statuto dei diritti del Contribuente), sono stati dichiarati i diritti del contribuente in occasione delle verifiche fiscali, di seguito riepilogati:
\begin{itemize}
    \item assistere personalmente a tutte le operazioni di verifica;
    \item farsi assistere da un professionista abilitato alla difesa dinanzi agli organi di giustizia tributaria;
    \item richiedere che l’esame dei documenti amministrativi e contabili sia effettuato negli uffici dei verificatori o presso il professionista che l’assiste o che la rappresenta;
    \item muovere rilievi o formulare osservazioni in relazione alle operazioni di controllo eseguite delle quali deve esserne dato atto nel verbale di verifica;
    \item rivolgersi al garante del contribuente, nei casi in cui ritenga che i verificatori stiano procedendo con modalità non conformi alla legge;
    \item rappresentare osservazioni e richieste all’Ufficio, entro 60 giorni dalla notifica del processo verbale di constatazione redatto a conclusione dell’intervento;
    \item esercitare ogni altro diritto previsto dalla legislazione tributaria vigente.
\end{itemize}
La informavano, inoltre, dei doveri del contribuente durante la verifica:
\begin{itemize}
    \item di consentire l’accesso nei locali destinati alla propria attività;
    \item di esibire, a richiesta dei verificatori, libri, registri, scritture e documenti;
    \item di presentare i prodotti ad ogni richiesta ed a sottoporsi a controlli o accertamenti;
    \item di consentire l’ispezione alle scritture contabili e dei documenti la cui tenuta e conservazione sono obbligatorie per legge o dei quali risulta l’esistenza;
    \item secondo quanto disposto dall’art. 52 del D.P.R. n. 633/1972, così come richiamato dall’art. 15 comma 8-duodecies del D.L. n. 78/2009 conv. L. n. 102/2009, i libri, i registri, le scritture e i documenti di cui venga rifiutata l’esibizione non potranno essere presi in considerazione a favore della parte, ai fini dell’accertamento in sede amministrativa e contenziosa; per rifiuto di esibizione si intendono anche le dichiarazioni di non possedere libri, registri, documenti e scritture e/o la sottrazione di essi al controllo;
    \item rifiutare l’esibizione o comunque impedire l’ispezione delle scritture contabili e dei documenti la cui tenuta e conservazione sono obbligatorie per legge o delle quali risulta l’esistenza, determina l’applicabilità della sanzione prevista dall’art. 11 comma 1 lett. c) del D.lgs. n. 471/1997, da € 258,00 a € 2065,00, qualora il fatto non costituisca più grave reato.
\end{itemize}
Riguardo alla facoltà di farsi assistere da un professionista abilitato alla difesa avanti la giurisdizione tributaria, la parte dichiara:
\\ \begin{math}\square\end{math} CHE NON INTENDE AVVALERSENE
\\ \begin{math}\square\end{math} CHE INTENDE AVVALERSENE, nella persona del Sig./Dr./Avv. \_\_\_\_\_\_\_\_\_\_\_\_\_\_\_\_\_\_\_\_\_\_\_\_\_\_\_\_\_\_\_\_\_\_\_\_\_\_\_\_\_\_\_\_ \\\_\_\_\_\_\_\_\_\_\_\_\_\_\_\_\_\_\_\_\_\_\_\_\_\_\_\_\_\_\_\_\_\_\_\_\_\_\_\_\_\_\_\_\_\_\_\_\_\_\_\_\_\_\_\_\_\_\_\_\_\_\_\_\_\_\_\_\_\_\_\_\_\_\_\_ che contestualmente viene avvisato telefonicamente dalla parte, e che interviene alle ore \_\_\_\_\_.\_\_\_\_\_.


\centering\subsection*{FATTO}

I funzionari in premessa indicati, durante la verifica hanno effettuato i sottoelencati controlli, dai quali sono emerse le risultanze di fianco a ciascuno indicate:
\begin{itemize}
    \item Autorizzazione vendita P.L.I. prot. n. \_\_\_\_\_\_\_\_\_\_\_\_\_\_\_\_\_ del \_\_\_\_\_/\_\_\_\_\_/\_\_\_\_\_\_\_\_\_
    \item Licenza/SCIA ai sensi dell'art. 86 del T.U.L.P.S. presentata al Comune di \_\_\_\_\_\_\_\_\_\_\_\_\_\_\_\_\_ in data \_\_\_\_\_/\_\_\_\_\_/\_\_\_\_\_\_\_\_\_ per "\_\_\_\_\_\_\_\_\_\_\_\_\_\_\_\_\_\_\_\_\_\_\_\_\_\_\_\_\_\_\_\_\_\_\_\_\_\_\_\_\_\_\_\_\_\_\_\_\_\_\_\_\_\_\_\_\_\_\_\_\_\_\_\_\_\_\_\_"
\end{itemize}

\begin{tabularx}{\textwidth}{|p{15cm}|Y|Y|}
    \hline
    a) Il patentino ordinario o speciale è regolarmente funzionante nei locali autorizzati dall'Agenzia & SÌ & NO \\
    \hline
    b) I generi di monopolio sono custoditi e venduti esclusivamente nei locali e nei luoghi per i quali il patentino ordinario o speciale è stato autorizzato & SÌ & NO \\
    \hline
    c) I generi di monopolio vengono venduti al pubblico ai prezzi stabiliti in tariffa senza alcuna variazione & SÌ & NO \\
    \hline
    d) Il personale addetto alla vendita dei generi di monopolio è regolarmente autorizzato & SÌ & NO \\
    \hline
    e) I generi di monopolio vengono prelevati, con regolare bolletta U-88 PAT, dalla rivendita di aggregazione prescritta dall'Agenzia & SÌ & NO \\
    \hline
    f) Il patentino ordinario o speciale è rifornito di tabacchi lavorati in quantità adeguata alle esigenze di consumo & SÌ & NO \\
    \hline
    g) La mostra dei generi di monopolio è adeguata e garantisce la neutralità dell'offerta & SÌ & NO \\
    \hline
    h) La vendita dei tabacchi non è associata o condizionata dalla consegna all'acquirente di gadget e omaggi e non sono esposti prodotti, materiali od oggettistica di qualsiasi genere che, anche in modo indiretto, costituiscano richiamo a qualsiasi marchio o prodotto da fumo & SÌ & NO \\
    \hline
    i) È tenuta in vista all'interno dell'esercizio l'autorizzazione alla vendita dei generi di monopolio & SÌ & NO \\
    \hline
    l) È tenuta in vista all'interno dell'esercizio la tariffa dei prezzi dei generi di monopolio aggiornata ovvero ne è comunque assicurata l'agevole ostensione & SÌ & NO \\
    \hline
    m) I generi di monopolio sono venduti nei condizionamenti usuali presenti in commercio ed autorizzati dall'Amministrazione & SÌ & NO \\
    \hline
    n) Il patentino ordinario o speciale non espone all'esterno o all'interno dell'esercizio scritte o insegne che indicano, anche indirettamente, la vendita di tabacchi & SÌ & NO \\
    \hline
    o) È regolarmente esposto il cartello di divieto della vendita tabacchi ai minori di anni 18 & SÌ & NO \\
    \hline
    p) All'atto dell'accesso non è stata riscontrata la vendita di tabacchi o sigarette elettroniche o contenitori di liquido di ricarica o prodotti del tabacco di nuova generazione o melassa per narghilè o tabacco per pipa ad acqua a minori di anni 18  & SÌ & NO \\
    \hline

\end{tabularx}

\newpage

In ottemperanza alle circolari, prot. n. DAC/CTL/11405/2012 del 26/07/2012 e prot. n. DAC/DIR/868/2012 del 16/11/2012, i verbalizzanti hanno effettuato le sotto indicate prove, finalizzate ad accertare l’autenticità dei contrassegni di Stato di cui al D. D. 23/06/2011, utilizzando il lettore \textbf{HD20}, matricola \textbf{PAG 521}:

\begin{tabularx}{\textwidth}{|p{.15\textwidth}|p{.45\textwidth}|p{.1\textwidth}|Y|}
    \hline
    Codice & Denominazione & n. pacchetti & Esito (regolare, irregolare, impossibile per presenza vecchio contrassegno)  \\
    \hline
     & & & \\[20pt]
    \hline
     & & & \\[20pt]
    \hline
     & & & \\[20pt]
    \hline
     & & & \\[20pt]
    \hline
     & & & \\[20pt]
    \hline
     & & & \\[20pt]
    \hline
\end{tabularx}

Vista l’irregolarità riscontrata su n. \_\_\_\_\_\_\_\_\_\_\_\_ confezioni di \_\_\_\_\_\_\_\_\_\_\_\_\_\_\_\_\_\_\_\_\_\_\_\_\_\_\_\_\_\_\_\_\_\_\_\_\_\_\_\_\_\_\_\_\_\_\_\_\_\_\_\_\_ marca \_\_\_\_\_\_\_\_\_\_\_\_\_\_\_\_\_\_\_\_\_\_\_\_\_\_\_\_\_\_\_\_\_\_\_\_\_\_\_\_\_\_\_\_\_\_\_\_ codice  \_\_\_\_\_\_\_\_\_\_\_\_\_\_\_\_\_\_\_\_\_\_\_\_ i verbalizzanti hanno provveduto a dare immediata comunicazione  del fatto alla Guardia di Finanza con richiesta di intervento tramite il n. 117.  In esito alla richiesta la Guardia di Finanza \\
\begin{math}\square\end{math} È \begin{math}\square\end{math} NON È\\
intervenuta ed ha emesso apposito verbale  di \_\_\_\_\_\_\_\_\_\_\_\_\_\_\_\_\_\_\_\_\_\_\_\_\_\_\_\_\_\_\_\_\_\_\_\_\_\_\_\_\_\_\_\_\_\_\_\_\_\_\_\_\_\_\_\_\_\_\_\_\_\_\_\_\_\_\_\_\_\_\_\_\_\_.

\paragraph{LOTTERIE AD ESTRAZIONE ISTANTANEA}
Ai fini delle attività di controllo, previste dall’\textbf{articolo 25} della \textit{Convenzione per il rapporto di concessione dell’esercizio dei giochi pubblici denominati lotterie nazionali ad estrazione istantanea} nonché dall’\textbf{articolo 11} del \textit{Contratto di autorizzazione all’esercizio dell’attività di vendita dei biglietti delle lotterie nazionali ad estrazione istantanee}, i verbalizzanti
\begin{math}\square\end{math} HANNO \begin{math}\square\end{math} NON HANNO
constatato che all’interno dei locali del predetto esercizio sono posti in vendita biglietti delle \textbf{lotterie ad estrazione istantanea}.\\
L'esame visivo dei suddetti biglietti posti in vendita risulta:
\begin{itemize}[label={\begin{math}\square\end{math}}]
    \item REGOLARE: biglietti integri, non contraffatti, relativi a concorsi autorizzati e non dichiarati cessati;
    \item NON REGOLARE: viene redatto il \textit{Verbale A)} di constatazione (allegato);\\ viene redatto il \textit{Verbale B)} di constatazione e comunicazione di notizia di reato ai sensi dell'art. 331 c.p.p. (allegato).\\
    Nel caso di riscontrata irregolarità, si ordina al titolare del punto vendita di ritirare immediatamente dalla vendita i biglietti di cui all’allegato Verbale A) – B) e custodirli al fine di consentire l’effettuazione delle necessarie verifiche da parte del concessionario e/o dell’organo giudiziario competente.
\end{itemize}

\paragraph{GIOCHI NUMERICI A TOTALIZZATORE NAZIONALE}
In ottemperanza alla circolare prot. n. 19039 del 12/02/2018 è stata controllata l’esistenza di terminali per giochi numerici a totalizzatore nazionale (GNTN) ed efficienza operativa degli stessi:
\begin{multicols}{2}
    \begin{itemize}
        \item terminali presenti: \begin{math}\square\end{math} SI \begin{math}\square\end{math} NO
        \item terminali operativi: \begin{math}\square\end{math} SI \begin{math}\square\end{math} NO
    \end{itemize}
\end{multicols}

Le eventuali irregolarità sopra accertate costituiscono violazione sanzionabile a norma dell’art. 7 della Legge 8 novembre 2012 n. 189 e della Legge n. 1293/1957 che saranno oggetto di successiva contestazione nei termini previsti dalle disposizioni vigenti.
\begin{multicols}{2}
    \begin{math}\square\end{math} REGOLARE \\
    \begin{math}\square\end{math} IRREGOLARE
\end{multicols}








\begin{center}
    \textbf{Dichiarazioni della parte}
\end{center}

\_\_\_\_\_\_\_\_\_\_\_\_\_\_\_\_\_\_\_\_\_\_\_\_\_\_\_\_\_\_\_\_\_\_\_\_\_\_\_\_\_\_\_\_\_\_\_\_\_\_\_\_\_\_\_\_\_\_\_\_\_\_\_\_\_\_\_\_\_\_\_\_\_\_\_\_\_\_\_\_\_\_\_\_\_\_\_\_\_\_\_\_\_\_\_\_\_\_\_\_\_\_\_\_\_\_\_\_\_\_\_\_\_\_\_\_\_\_\_\_\_\_\_\_\_\_\_ \\
\_\_\_\_\_\_\_\_\_\_\_\_\_\_\_\_\_\_\_\_\_\_\_\_\_\_\_\_\_\_\_\_\_\_\_\_\_\_\_\_\_\_\_\_\_\_\_\_\_\_\_\_\_\_\_\_\_\_\_\_\_\_\_\_\_\_\_\_\_\_\_\_\_\_\_\_\_\_\_\_\_\_\_\_\_\_\_\_\_\_\_\_\_\_\_\_\_\_\_\_\_\_\_\_\_\_\_\_\_\_\_\_\_\_\_\_\_\_\_\_\_\_\_\_\_\_\_ \\\_\_\_\_\_\_\_\_\_\_\_\_\_\_\_\_\_\_\_\_\_\_\_\_\_\_\_\_\_\_\_\_\_\_\_\_\_\_\_\_\_\_\_\_\_\_\_\_\_\_\_\_\_\_\_\_\_\_\_\_\_\_\_\_\_\_\_\_\_\_\_\_\_\_\_\_\_\_\_\_\_\_\_\_\_\_\_\_\_\_\_\_\_\_\_\_\_\_\_\_\_\_\_\_\_\_\_\_\_\_\_\_\_\_\_\_\_\_\_\_\_\_\_\_\_\_\_ \\\_\_\_\_\_\_\_\_\_\_\_\_\_\_\_\_\_\_\_\_\_\_\_\_\_\_\_\_\_\_\_\_\_\_\_\_\_\_\_\_\_\_\_\_\_\_\_\_\_\_\_\_\_\_\_\_\_\_\_\_\_\_\_\_\_\_\_\_\_\_\_\_\_\_\_\_\_\_\_\_\_\_\_\_\_\_\_\_\_\_\_\_\_\_\_\_\_\_\_\_\_\_\_\_\_\_\_\_\_\_\_\_\_\_\_\_\_\_\_\_\_\_\_\_\_\_\_ \\\_\_\_\_\_\_\_\_\_\_\_\_\_\_\_\_\_\_\_\_\_\_\_\_\_\_\_\_\_\_\_\_\_\_\_\_\_\_\_\_\_\_\_\_\_\_\_\_\_\_\_\_\_\_\_\_\_\_\_\_\_\_\_\_\_\_\_\_\_\_\_\_\_\_\_\_\_\_\_\_\_\_\_\_\_\_\_\_\_\_\_\_\_\_\_\_\_\_\_\_\_\_\_\_\_\_\_\_\_\_\_\_\_\_\_\_\_\_\_\_\_\_\_\_\_\_\_ \\\_\_\_\_\_\_\_\_\_\_\_\_\_\_\_\_\_\_\_\_\_\_\_\_\_\_\_\_\_\_\_\_\_\_\_\_\_\_\_\_\_\_\_\_\_\_\_\_\_\_\_\_\_\_\_\_\_\_\_\_\_\_\_\_\_\_\_\_\_\_\_\_\_\_\_\_\_\_\_\_\_\_\_\_\_\_\_\_\_\_\_\_\_\_\_\_\_\_\_\_\_\_\_\_\_\_\_\_\_\_\_\_\_\_\_\_\_\_\_\_\_\_\_\_\_\_\_ \\\_\_\_\_\_\_\_\_\_\_\_\_\_\_\_\_\_\_\_\_\_\_\_\_\_\_\_\_\_\_\_\_\_\_\_\_\_\_\_\_\_\_\_\_\_\_\_\_\_\_\_\_\_\_\_\_\_\_\_\_\_\_\_\_\_\_\_\_\_\_\_\_\_\_\_\_\_\_\_\_\_\_\_\_\_\_\_\_\_\_\_\_\_\_\_\_\_\_\_\_\_\_\_\_\_\_\_\_\_\_\_\_\_\_\_\_\_\_\_\_\_\_\_\_\_\_\_ \\\_\_\_\_\_\_\_\_\_\_\_\_\_\_\_\_\_\_\_\_\_\_\_\_\_\_\_\_\_\_\_\_\_\_\_\_\_\_\_\_\_\_\_\_\_\_\_\_\_\_\_\_\_\_\_\_\_\_\_\_\_\_\_\_\_\_\_\_\_\_\_\_\_\_\_\_\_\_\_\_\_\_\_\_\_\_\_\_\_\_\_\_\_\_\_\_\_\_\_\_\_\_\_\_\_\_\_\_\_\_\_\_\_\_\_\_\_\_\_\_\_\_\_\_\_\_\_ \\\_\_\_\_\_\_\_\_\_\_\_\_\_\_\_\_\_\_\_\_\_\_\_\_\_\_\_\_\_\_\_\_\_\_\_\_\_\_\_\_\_\_\_\_\_\_\_\_\_\_\_\_\_\_\_\_\_\_\_\_\_\_\_\_\_\_\_\_\_\_\_\_\_\_\_\_\_\_\_\_\_\_\_\_\_\_\_\_\_\_\_\_\_\_\_\_\_\_\_\_\_\_\_\_\_\_\_\_\_\_\_\_\_\_\_\_\_\_\_\_\_\_\_\_\_\_\_ \\\_\_\_\_\_\_\_\_\_\_\_\_\_\_\_\_\_\_\_\_\_\_\_\_\_\_\_\_\_\_\_\_\_\_\_\_\_\_\_\_\_\_\_\_\_\_\_\_\_\_\_\_\_\_\_\_\_\_\_\_\_\_\_\_\_\_\_\_\_\_\_\_\_\_\_\_\_\_\_\_\_\_\_\_\_\_\_\_\_\_\_\_\_\_\_\_\_\_\_\_\_\_\_\_\_\_\_\_\_\_\_\_\_\_\_\_\_\_\_\_\_\_\_\_\_\_\_

\begin{center}
    \textbf{Dichiarazioni dei verbalizzanti}
\end{center}

\_\_\_\_\_\_\_\_\_\_\_\_\_\_\_\_\_\_\_\_\_\_\_\_\_\_\_\_\_\_\_\_\_\_\_\_\_\_\_\_\_\_\_\_\_\_\_\_\_\_\_\_\_\_\_\_\_\_\_\_\_\_\_\_\_\_\_\_\_\_\_\_\_\_\_\_\_\_\_\_\_\_\_\_\_\_\_\_\_\_\_\_\_\_\_\_\_\_\_\_\_\_\_\_\_\_\_\_\_\_\_\_\_\_\_\_\_\_\_\_\_\_\_\_\_\_\_ \\
\_\_\_\_\_\_\_\_\_\_\_\_\_\_\_\_\_\_\_\_\_\_\_\_\_\_\_\_\_\_\_\_\_\_\_\_\_\_\_\_\_\_\_\_\_\_\_\_\_\_\_\_\_\_\_\_\_\_\_\_\_\_\_\_\_\_\_\_\_\_\_\_\_\_\_\_\_\_\_\_\_\_\_\_\_\_\_\_\_\_\_\_\_\_\_\_\_\_\_\_\_\_\_\_\_\_\_\_\_\_\_\_\_\_\_\_\_\_\_\_\_\_\_\_\_\_\_ \\\_\_\_\_\_\_\_\_\_\_\_\_\_\_\_\_\_\_\_\_\_\_\_\_\_\_\_\_\_\_\_\_\_\_\_\_\_\_\_\_\_\_\_\_\_\_\_\_\_\_\_\_\_\_\_\_\_\_\_\_\_\_\_\_\_\_\_\_\_\_\_\_\_\_\_\_\_\_\_\_\_\_\_\_\_\_\_\_\_\_\_\_\_\_\_\_\_\_\_\_\_\_\_\_\_\_\_\_\_\_\_\_\_\_\_\_\_\_\_\_\_\_\_\_\_\_\_ \\\_\_\_\_\_\_\_\_\_\_\_\_\_\_\_\_\_\_\_\_\_\_\_\_\_\_\_\_\_\_\_\_\_\_\_\_\_\_\_\_\_\_\_\_\_\_\_\_\_\_\_\_\_\_\_\_\_\_\_\_\_\_\_\_\_\_\_\_\_\_\_\_\_\_\_\_\_\_\_\_\_\_\_\_\_\_\_\_\_\_\_\_\_\_\_\_\_\_\_\_\_\_\_\_\_\_\_\_\_\_\_\_\_\_\_\_\_\_\_\_\_\_\_\_\_\_\_ \\\_\_\_\_\_\_\_\_\_\_\_\_\_\_\_\_\_\_\_\_\_\_\_\_\_\_\_\_\_\_\_\_\_\_\_\_\_\_\_\_\_\_\_\_\_\_\_\_\_\_\_\_\_\_\_\_\_\_\_\_\_\_\_\_\_\_\_\_\_\_\_\_\_\_\_\_\_\_\_\_\_\_\_\_\_\_\_\_\_\_\_\_\_\_\_\_\_\_\_\_\_\_\_\_\_\_\_\_\_\_\_\_\_\_\_\_\_\_\_\_\_\_\_\_\_\_\_ \\\_\_\_\_\_\_\_\_\_\_\_\_\_\_\_\_\_\_\_\_\_\_\_\_\_\_\_\_\_\_\_\_\_\_\_\_\_\_\_\_\_\_\_\_\_\_\_\_\_\_\_\_\_\_\_\_\_\_\_\_\_\_\_\_\_\_\_\_\_\_\_\_\_\_\_\_\_\_\_\_\_\_\_\_\_\_\_\_\_\_\_\_\_\_\_\_\_\_\_\_\_\_\_\_\_\_\_\_\_\_\_\_\_\_\_\_\_\_\_\_\_\_\_\_\_\_\_ \\\_\_\_\_\_\_\_\_\_\_\_\_\_\_\_\_\_\_\_\_\_\_\_\_\_\_\_\_\_\_\_\_\_\_\_\_\_\_\_\_\_\_\_\_\_\_\_\_\_\_\_\_\_\_\_\_\_\_\_\_\_\_\_\_\_\_\_\_\_\_\_\_\_\_\_\_\_\_\_\_\_\_\_\_\_\_\_\_\_\_\_\_\_\_\_\_\_\_\_\_\_\_\_\_\_\_\_\_\_\_\_\_\_\_\_\_\_\_\_\_\_\_\_\_\_\_\_ \\\_\_\_\_\_\_\_\_\_\_\_\_\_\_\_\_\_\_\_\_\_\_\_\_\_\_\_\_\_\_\_\_\_\_\_\_\_\_\_\_\_\_\_\_\_\_\_\_\_\_\_\_\_\_\_\_\_\_\_\_\_\_\_\_\_\_\_\_\_\_\_\_\_\_\_\_\_\_\_\_\_\_\_\_\_\_\_\_\_\_\_\_\_\_\_\_\_\_\_\_\_\_\_\_\_\_\_\_\_\_\_\_\_\_\_\_\_\_\_\_\_\_\_\_\_\_\_ \\\_\_\_\_\_\_\_\_\_\_\_\_\_\_\_\_\_\_\_\_\_\_\_\_\_\_\_\_\_\_\_\_\_\_\_\_\_\_\_\_\_\_\_\_\_\_\_\_\_\_\_\_\_\_\_\_\_\_\_\_\_\_\_\_\_\_\_\_\_\_\_\_\_\_\_\_\_\_\_\_\_\_\_\_\_\_\_\_\_\_\_\_\_\_\_\_\_\_\_\_\_\_\_\_\_\_\_\_\_\_\_\_\_\_\_\_\_\_\_\_\_\_\_\_\_\_\_ \\\_\_\_\_\_\_\_\_\_\_\_\_\_\_\_\_\_\_\_\_\_\_\_\_\_\_\_\_\_\_\_\_\_\_\_\_\_\_\_\_\_\_\_\_\_\_\_\_\_\_\_\_\_\_\_\_\_\_\_\_\_\_\_\_\_\_\_\_\_\_\_\_\_\_\_\_\_\_\_\_\_\_\_\_\_\_\_\_\_\_\_\_\_\_\_\_\_\_\_\_\_\_\_\_\_\_\_\_\_\_\_\_\_\_\_\_\_\_\_\_\_\_\_\_\_\_\_ \\\_\_\_\_\_\_\_\_\_\_\_\_\_\_\_\_\_\_\_\_\_\_\_\_\_\_\_\_\_\_\_\_\_\_\_\_\_\_\_\_\_\_\_\_\_\_\_\_\_\_\_\_\_\_\_\_\_\_\_\_\_\_\_\_\_\_\_\_\_\_\_\_\_\_\_\_\_\_\_\_\_\_\_\_\_\_\_\_\_\_\_\_\_\_\_\_\_\_\_\_\_\_\_\_\_\_\_\_\_\_\_\_\_\_\_\_\_\_\_\_\_\_\_\_\_\_\_ \\\_\_\_\_\_\_\_\_\_\_\_\_\_\_\_\_\_\_\_\_\_\_\_\_\_\_\_\_\_\_\_\_\_\_\_\_\_\_\_\_\_\_\_\_\_\_\_\_\_\_\_\_\_\_\_\_\_\_\_\_\_\_\_\_\_\_\_\_\_\_\_\_\_\_\_\_\_\_\_\_\_\_\_\_\_\_\_\_\_\_\_\_\_\_\_\_\_\_\_\_\_\_\_\_\_\_\_\_\_\_\_\_\_\_\_\_\_\_\_\_\_\_\_\_\_\_\_ \\\_\_\_\_\_\_\_\_\_\_\_\_\_\_\_\_\_\_\_\_\_\_\_\_\_\_\_\_\_\_\_\_\_\_\_\_\_\_\_\_\_\_\_\_\_\_\_\_\_\_\_\_\_\_\_\_\_\_\_\_\_\_\_\_\_\_\_\_\_\_\_\_\_\_\_\_\_\_\_\_\_\_\_\_\_\_\_\_\_\_\_\_\_\_\_\_\_\_\_\_\_\_\_\_\_\_\_\_\_\_\_\_\_\_\_\_\_\_\_\_\_\_\_\_\_\_\_ \\\_\_\_\_\_\_\_\_\_\_\_\_\_\_\_\_\_\_\_\_\_\_\_\_\_\_\_\_\_\_\_\_\_\_\_\_\_\_\_\_\_\_\_\_\_\_\_\_\_\_\_\_\_\_\_\_\_\_\_\_\_\_\_\_\_\_\_\_\_\_\_\_\_\_\_\_\_\_\_\_\_\_\_\_\_\_\_\_\_\_\_\_\_\_\_\_\_\_\_\_\_\_\_\_\_\_\_\_\_\_\_\_\_\_\_\_\_\_\_\_\_\_\_\_\_\_\_ \\\_\_\_\_\_\_\_\_\_\_\_\_\_\_\_\_\_\_\_\_\_\_\_\_\_\_\_\_\_\_\_\_\_\_\_\_\_\_\_\_\_\_\_\_\_\_\_\_\_\_\_\_\_\_\_\_\_\_\_\_\_\_\_\_\_\_\_\_\_\_\_\_\_\_\_\_\_\_\_\_\_\_\_\_\_\_\_\_\_\_\_\_\_\_\_\_\_\_\_\_\_\_\_\_\_\_\_\_\_\_\_\_\_\_\_\_\_\_\_\_\_\_\_\_\_\_\_
\\\_\_\_\_\_\_\_\_\_\_\_\_\_\_\_\_\_\_\_\_\_\_\_\_\_\_\_\_\_\_\_\_\_\_\_\_\_\_\_\_\_\_\_\_\_\_\_\_\_\_\_\_\_\_\_\_\_\_\_\_\_\_\_\_\_\_\_\_\_\_\_\_\_\_\_\_\_\_\_\_\_\_\_\_\_\_\_\_\_\_\_\_\_\_\_\_\_\_\_\_\_\_\_\_\_\_\_\_\_\_\_\_\_\_\_\_\_\_\_\_\_\_\_\_\_\_\_
\\\_\_\_\_\_\_\_\_\_\_\_\_\_\_\_\_\_\_\_\_\_\_\_\_\_\_\_\_\_\_\_\_\_\_\_\_\_\_\_\_\_\_\_\_\_\_\_\_\_\_\_\_\_\_\_\_\_\_\_\_\_\_\_\_\_\_\_\_\_\_\_\_\_\_\_\_\_\_\_\_\_\_\_\_\_\_\_\_\_\_\_\_\_\_\_\_\_\_\_\_\_\_\_\_\_\_\_\_\_\_\_\_\_\_\_\_\_\_\_\_\_\_\_\_\_\_\_
\\\_\_\_\_\_\_\_\_\_\_\_\_\_\_\_\_\_\_\_\_\_\_\_\_\_\_\_\_\_\_\_\_\_\_\_\_\_\_\_\_\_\_\_\_\_\_\_\_\_\_\_\_\_\_\_\_\_\_\_\_\_\_\_\_\_\_\_\_\_\_\_\_\_\_\_\_\_\_\_\_\_\_\_\_\_\_\_\_\_\_\_\_\_\_\_\_\_\_\_\_\_\_\_\_\_\_\_\_\_\_\_\_\_\_\_\_\_\_\_\_\_\_\_\_\_\_\_
\\\_\_\_\_\_\_\_\_\_\_\_\_\_\_\_\_\_\_\_\_\_\_\_\_\_\_\_\_\_\_\_\_\_\_\_\_\_\_\_\_\_\_\_\_\_\_\_\_\_\_\_\_\_\_\_\_\_\_\_\_\_\_\_\_\_\_\_\_\_\_\_\_\_\_\_\_\_\_\_\_\_\_\_\_\_\_\_\_\_\_\_\_\_\_\_\_\_\_\_\_\_\_\_\_\_\_\_\_\_\_\_\_\_\_\_\_\_\_\_\_\_\_\_\_\_\_\_
\\\_\_\_\_\_\_\_\_\_\_\_\_\_\_\_\_\_\_\_\_\_\_\_\_\_\_\_\_\_\_\_\_\_\_\_\_\_\_\_\_\_\_\_\_\_\_\_\_\_\_\_\_\_\_\_\_\_\_\_\_\_\_\_\_\_\_\_\_\_\_\_\_\_\_\_\_\_\_\_\_\_\_\_\_\_\_\_\_\_\_\_\_\_\_\_\_\_\_\_\_\_\_\_\_\_\_\_\_\_\_\_\_\_\_\_\_\_\_\_\_\_\_\_\_\_\_\_
\\\_\_\_\_\_\_\_\_\_\_\_\_\_\_\_\_\_\_\_\_\_\_\_\_\_\_\_\_\_\_\_\_\_\_\_\_\_\_\_\_\_\_\_\_\_\_\_\_\_\_\_\_\_\_\_\_\_\_\_\_\_\_\_\_\_\_\_\_\_\_\_\_\_\_\_\_\_\_\_\_\_\_\_\_\_\_\_\_\_\_\_\_\_\_\_\_\_\_\_\_\_\_\_\_\_\_\_\_\_\_\_\_\_\_\_\_\_\_\_\_\_\_\_\_\_\_\_
\\\_\_\_\_\_\_\_\_\_\_\_\_\_\_\_\_\_\_\_\_\_\_\_\_\_\_\_\_\_\_\_\_\_\_\_\_\_\_\_\_\_\_\_\_\_\_\_\_\_\_\_\_\_\_\_\_\_\_\_\_\_\_\_\_\_\_\_\_\_\_\_\_\_\_\_\_\_\_\_\_\_\_\_\_\_\_\_\_\_\_\_\_\_\_\_\_\_\_\_\_\_\_\_\_\_\_\_\_\_\_\_\_\_\_\_\_\_\_\_\_\_\_\_\_\_\_\_
\\\_\_\_\_\_\_\_\_\_\_\_\_\_\_\_\_\_\_\_\_\_\_\_\_\_\_\_\_\_\_\_\_\_\_\_\_\_\_\_\_\_\_\_\_\_\_\_\_\_\_\_\_\_\_\_\_\_\_\_\_\_\_\_\_\_\_\_\_\_\_\_\_\_\_\_\_\_\_\_\_\_\_\_\_\_\_\_\_\_\_\_\_\_\_\_\_\_\_\_\_\_\_\_\_\_\_\_\_\_\_\_\_\_\_\_\_\_\_\_\_\_\_\_\_\_\_\_
\\\_\_\_\_\_\_\_\_\_\_\_\_\_\_\_\_\_\_\_\_\_\_\_\_\_\_\_\_\_\_\_\_\_\_\_\_\_\_\_\_\_\_\_\_\_\_\_\_\_\_\_\_\_\_\_\_\_\_\_\_\_\_\_\_\_\_\_\_\_\_\_\_\_\_\_\_\_\_\_\_\_\_\_\_\_\_\_\_\_\_\_\_\_\_\_\_\_\_\_\_\_\_\_\_\_\_\_\_\_\_\_\_\_\_\_\_\_\_\_\_\_\_\_\_\_\_\_
\\\_\_\_\_\_\_\_\_\_\_\_\_\_\_\_\_\_\_\_\_\_\_\_\_\_\_\_\_\_\_\_\_\_\_\_\_\_\_\_\_\_\_\_\_\_\_\_\_\_\_\_\_\_\_\_\_\_\_\_\_\_\_\_\_\_\_\_\_\_\_\_\_\_\_\_\_\_\_\_\_\_\_\_\_\_\_\_\_\_\_\_\_\_\_\_\_\_\_\_\_\_\_\_\_\_\_\_\_\_\_\_\_\_\_\_\_\_\_\_\_\_\_\_\_\_\_\_
\\\_\_\_\_\_\_\_\_\_\_\_\_\_\_\_\_\_\_\_\_\_\_\_\_\_\_\_\_\_\_\_\_\_\_\_\_\_\_\_\_\_\_\_\_\_\_\_\_\_\_\_\_\_\_\_\_\_\_\_\_\_\_\_\_\_\_\_\_\_\_\_\_\_\_\_\_\_\_\_\_\_\_\_\_\_\_\_\_\_\_\_\_\_\_\_\_\_\_\_\_\_\_\_\_\_\_\_\_\_\_\_\_\_\_\_\_\_\_\_\_\_\_\_\_\_\_\_\\
\_\_\_\_\_\_\_\_\_\_\_\_\_\_\_\_\_\_\_\_\_\_\_\_\_\_\_\_\_\_\_\_\_\_\_\_\_\_\_\_\_\_\_\_\_\_\_\_\_\_\_\_\_\_\_\_\_\_\_\_\_\_\_\_\_\_\_\_\_\_\_\_\_\_\_\_\_\_\_\_\_\_\_\_\_\_\_\_\_\_\_\_\_\_\_\_\_\_\_\_\_\_\_\_\_\_\_\_\_\_\_\_\_\_\_\_\_\_\_\_\_\_\_\_\_\_\_\\
\_\_\_\_\_\_\_\_\_\_\_\_\_\_\_\_\_\_\_\_\_\_\_\_\_\_\_\_\_\_\_\_\_\_\_\_\_\_\_\_\_\_\_\_\_\_\_\_\_\_\_\_\_\_\_\_\_\_\_\_\_\_\_\_\_\_\_\_\_\_\_\_\_\_\_\_\_\_\_\_\_\_\_\_\_\_\_\_\_\_\_\_\_\_\_\_\_\_\_\_\_\_\_\_\_\_\_\_\_\_\_\_\_\_\_\_\_\_\_\_\_\_\_\_\_\_\_\\
\_\_\_\_\_\_\_\_\_\_\_\_\_\_\_\_\_\_\_\_\_\_\_\_\_\_\_\_\_\_\_\_\_\_\_\_\_\_\_\_\_\_\_\_\_\_\_\_\_\_\_\_\_\_\_\_\_\_\_\_\_\_\_\_\_\_\_\_\_\_\_\_\_\_\_\_\_\_\_\_\_\_\_\_\_\_\_\_\_\_\_\_\_\_\_\_\_\_\_\_\_\_\_\_\_\_\_\_\_\_\_\_\_\_\_\_\_\_\_\_\_\_\_\_\_\_\_\\

\textit{\textbf{Per le eventuali irregolarità l'Ufficio si riserva ogni doverosa e opportuna azione amministrativa con successivi atti.}}


Si dà atto che durante le operazioni di verifica, svoltesi con la continua assistenza della parte, non sono stati arrecati danni né agli apparecchi oggetto di controllo né ai beni mobili e immobili, che nulla è stato asportato e che la parte non ha nulla da lamentare sull’operato dei verbalizzanti.\\
Si dà atto, altresì, che le operazioni si sono protratte per il tempo strettamente necessario allo svolgimento delle stesse, ed hanno avuto termine alle ore \_\_\_\_\_\_.\_\_\_\_\_\_ dello stesso giorno.\\
Il presente atto, che si compone di n. 7 facciate e n° \_\_\_\_\_\_ allegati, è redatto in n° 2 originali, uno per l’Ufficio dei Monopoli per le Marche dell’Agenzia delle Dogane e dei Monopoli - Sede di Pesaro, ed uno per la parte.\\
Il verbale, letto e confermato senza ulteriori osservazioni ed eccezione di sorta, viene sottoscritto dai funzionari verbalizzanti e dalla parte, a cui si rilascia copia.\\


Allegati:
\begin{itemize}[label={}]
    \item \begin{math}\square\end{math} Allegato A \begin{math}\square\end{math} Allegato B \begin{math}\square\end{math} Allegato C \begin{math}\square\end{math} Documentazione fotografica
    \item \begin{math}\square\end{math} Letture \begin{math}\square\end{math} Altro \_\_\_\_\_\_\_\_\_\_\_\_\_\_\_\_\_\_\_\_\_\_\_\_\_\_\_\_\_\_\_\_\_\_\_\_\_\_\_\_\_\_\_\_\_\_\_\_\_\_\_\_\_\_\_\_\_\_\_\_
\end{itemize}

Recapiti per l'invio di eventuale documentazione:
\begin{itemize}[label={}]
    \item EMAIL: monopoli.marche.urp@adm.gov.it
    \item PEC: monopoli.ancona@pec.adm.gov.it
\end{itemize}

\signature


\end{document}